\defmodule {fdist}

This module provides procedures to compute (or approximate)
the distribution functions of several standard types of random variables
and of certain goodness-of-fit test statistics.
Recall that the distribution function of a continuous random variable $X$ 
with density $f$ is
\eq
  F(x) = P[X\le x] = \int_{-\infty}^x f(x)dx 
\endeq
while that of a discrete random variable $X$ with mass function $f$ 
over the set of integers is
\eq
  F(x) = P[X\le x] = \sum_{s=-\infty}^x f(x).
\endeq
All the procedures in this module return $F(x)$ for some 
probability distribution.

Most distributions are implemented only in standardized form here,
i.e., with the location parameter set to 0 and the scale parameter
set to 1.  To shift the distribution by $x_0$ and rescale by $c$, 
it suffices to replace $x$ by $(x-x_0)/c$ in the argument when 
calling the function.

%\begin{detailed}
 For some of the discrete distributions, the value of $F(x)$ can be
 simply recovered from a table that would have been previously
 constructed; see the module {\tt fmass} for the details.
 This permits one to avoid recomputing the sums.
%\end{detailed}


%%%%%%%%%%%%%%%%%%%%%%%%%%%%%%%%%%%%%%%%%%
\bigskip\hrule\medskip
\code\hide
/* fdist.h for ANSI C */
#ifndef FDIST_H
#define FDIST_H
\endhide
#include "gdef.h"
#include "fmass.h"
\endcode

%%%%%%%%%%%%%%%%%%%%%%%%%%%%%%%%%%%%%%%%%%
\guisec{Continuous distributions}

\code

double fdist_Unif (double x);
\endcode
  \tab
  Returns $x$ for $x \in [0, 1]$, returns 0 for $x < 0$, and returns 1
  for $x > 1$. This is the uniform distribution function over $[0, 1]$.
 \endtab
\code


double fdist_Expon (double x);
\endcode
 \tab
  Returns 
  \eq
   F(x) = 1 - e^{- x}                          \eqlabel{eq:Fexpon}
  \endeq
  for $x > 0$, and 0 for $x<0$.  This is the standard exponential 
  distribution \cite{tJOH95a} with mean 1.
 \endtab
\code


double fdist_Weibull (double alpha, double x);
\endcode
  \tab
  Returns 
  \eq
   F(x) = 1 - e^{- x^\alpha},                 \eqlabel{eq:Fweibull}
  \endeq
  for $x>0$, and 0 for $x\le 0$.
  This is the standard Weibull distribution function \cite{tJOH95a} with shape 
  parameter $\alpha$.
  Restriction: $\alpha > 0$.
 \hpierre {D'habitude la Weibull a 2 ou 3 param\'etres. Il faudrait 
   peut-etre ajouter les autres param\`etres.  M\^eme chose pour les autres
   loi de probabilit\'es qui suivent? }
 \endtab
\code


double fdist_ExtremeValue (double x);
\endcode
 \tab
  Returns 
 \eq
     F(x) = e^{-e^{-x}},                       \eqlabel{eq:Fextreme}
 \endeq
  the standard extreme value distribution function \cite{tJOH95b}.
 \endtab
\code


double fdist_Logistic (double x);
\endcode
 \tab
  Returns 
  \eq F(x) = \frac 1{1 + e^{-x}} =
       \frac12\left(1 + \tanh \left(\frac{x}{2}\right)\right),
                                              \eqlabel{eq:Flogistic}
  \endeq
  the standard logistic distribution function \cite{tJOH95b}.
 \endtab
\code


double fdist_Pareto (double c, double x);
\endcode
  \tab
  Returns
  \eq
     F(x) = 1 - \frac 1 {x^c},
                                              \eqlabel{eq:Fpareto}
  \endeq
  for $x\ge 1$ and 0 for $x<1$.  This is
  the standard Pareto distribution function \cite{tJOH95a}.
  Restriction: $c > 0$.
 \endtab
\code


double fdist_Normal1 (double x);
\endcode
  \tab  
  Returns an approximation of $\Phi(x)$, where $\Phi$ is the standard normal
  distribution function, with mean 0 and variance 1. 
  Uses the approximation given in \cite[page 90]{tKEN80a}. This distribution
  is less precise than {\tt fdist\_Normal2} in the lower tail, as it will
  not compute probabilities smaller than {\tt DBL\_EPSILON}.
 \endtab
\code


double fdist_Normal2 (double x);
\endcode
  \tab  
  Returns an approximation of $\Phi(x)$, 
  where $\Phi$ is the standard normal distribution function,
  with mean 0 and variance 1. 
  Uses the Chebyshev approximation proposed in \cite{tSCH78a},
  which gives 15 decimals of precision nearly everywhere.
%  In the paper, the author gives the coefficients with 30 decimals of
%  precision.   
  This function is 1.5 times slower than {\tt fdist\_Normal1}.
 \endtab
\code


#ifdef HAVE_ERF
   double fdist_Normal3 (double x);
#endif
\endcode
  \tab
 \hpierre{Je trouve ces {\tt ifdef} tr\`es laids.  Pas moyen de cacher
    cela d'une certaine mani\`ere?   De plus {\tt HAVE\_ERF} n'est pas
    d\'efini ni expliqu\'e. }
\hrichard {Toutes les macros HAVE OU USE proviennent du module gdef de mylib.
Si on veut que l'utilisateur puisse les utiliser ou non, d\'ependant
de sa machine, il faut qu'il les voit; sinon, il faudra l'\'eliminer.}
  Returns an approximation of $\Phi(x)$, 
  where $\Phi$ is the standard normal distribution function,
  with mean 0 and variance 1. 
  Uses the {\tt erf} function from the standard Unix C library. The macro
  {\tt HAVE\_ERF} from {\tt mylib/gdef} must be defined. On some machines,
   this function is twice as fast as {\tt fdist\_Normal1}.
 \endtab
\code


double fdist_Normal4 (double x);
\endcode
  \tab  
  Returns an approximation of $\Phi(x)$,  where $\Phi$ is the standard
  normal distribution function, with mean 0 and variance 1. 
   Uses Marsaglia's et al \cite{rMAR94b} fast method
  with tables lookup. Returns 15 decimal digits of precision.
  This function is as fast as {\tt fdist\_Normal1} (no more no less).
 \endtab
\code


double fdist_BiNormal1 (double x, double y, double rho, int ndig);
\endcode
  \tab  
  Returns the value $u$ of the standard bivariate normal distribution,
  given by
\begin{eqnarray}
     u &=&  \frac{1}{2\pi\sqrt{1 - \rho^2}} \int_{-\infty}^x
              \int_{-\infty}^y e^{-T} dy\, dx  \label{eq.binormal1} \\[5pt]
     T &=& \frac{x^2 -2\rho x y + y^2}{2(1-\rho^2)}, \nonumber
\end{eqnarray}
  where $\rho = ${ \tt rho} is the correlation between $x$ and $y$, and
 \texttt{ndig} is the number of decimal digits of accuracy.
  The code was translated from the Fortran program written
   by T. G. Donnelly \cite{tDON73a} and copyrighted by the ACM (see 
  \url{http://www.acm.org/pubs/copyright_policy/#Notice}). The absolute error
  is expected to be smaller than $10^{-d}$, where $d={}$\texttt{ndig}.
  Restriction: \texttt{ndig}${} \le 15$.
 \endtab
\code


double fdist_BiNormal2 (double x, double y, double rho);
\endcode
  \tab  
  Returns the value of the standard bivariate normal distribution as 
  defined in (\ref{eq.binormal1}) above.
  It was translated directly from the Matlab code written by Alan Genz
  and available from his web page (the code is copyrighted by Alan Genz, 
  and is included in this package with the kind permission of its author).
  The algorithm, described in \cite{tGEN04a}, is a modified form of the 
  algorithm proposed in \cite{tDRE89a}. The program's accuracy results
  in an absolute error less than $5 \cdot 10^{-16}$. 
 \endtab
\code


double fdist_LogNormal (double mu, double sigma, double x);
\endcode
  \tab
  Returns the lognormal distribution function, defined by \cite{tJOH95a}
  \eq
     F(x) = \Phi \left(\frac{\ln (x) - \mu}{\sigma}\right)
  \endeq
  for $x>0$ and 0 for $x\le 0$,
  where $\Phi$ is the standard normal distribution.
  Restriction: $\sigma > 0$.
 \endtab
\code


double fdist_JohnsonSB (double alpha, double beta, double a, double b,
                        double x);
\endcode
  \tab
  Returns the Johnson JSB distribution function \cite{sLAW00a}:
 \eq
   F(x) = \Phi\left(\alpha + \beta\ln\left(\frac{x-a}{b-x} \right)\right),
 \endeq
  where $\Phi$ is the standard normal distribution.
  Restrictions: $\beta>0$, $a < b$, and $a \le x \le b$.
 \endtab
\code


double fdist_JohnsonSU (double alpha, double beta, double x);
\endcode
  \tab
  Returns the Johnson JSU distribution function \cite{sLAW00a}:
  \eq
   F(x) = \Phi\left(\alpha + \beta\ln\left(x +
          \sqrt{x^2 + 1} \right)\right) 
  \endeq
  where $\Phi$ is the standard normal distribution.
  Restriction: $\beta>0$.
 \endtab
\code


double fdist_ChiSquare1 (long k, double x);
\endcode
  \tab
  Returns an approximation of the chi-square distribution function 
  with $k$ degrees of freedom, 
\iffalse 
 whose density is
 \eq
   f(x) =                       \eqlabel{eq:Fchi2}
 \endeq
  for $x\ge 0$.
\fi
  which is a special case of the gamma distribution, with shape parameter
  $k/2$ and scale parameter $1/2$.
  Uses the approximation given in \cite[p.116]{tKEN80a} for $k\le 1000$,
  and the normal approximation for $k > 1000$. Gives no more than 4
  decimals of precision for $k > 1000$.
 \endtab
\code


double fdist_ChiSquare2 (long k, int d, double x);
\endcode
  \tab
  Returns an approximation of the chi-square distribution function 
  with $k$ degrees of freedom, by calling {\tt fdist\_Gamma (k/2, d, x/2)}.
  The function will do its best to return $d$ decimals
  digits of precision (but there is no guarantee).
  For $k$ not too large (e.g., $k \le 1000$),
  $d$ gives a good idea of the precision attained.
  Restrictions:  $k>0$ and $0 < d \le 15$.
 \endtab
\code


double fdist_Student1 (long n, double x);
\endcode
  \tab
  Returns the approximation of \cite[p.96]{tKEN80a} for the 
  {\em Student-$t$\/} distribution function with $n$ degrees of freedom,
  whose density is
   \eq
        f(x) = \frac{\Gamma\left((n + 1)/2 \right)}
                        {\Gamma(n/2) \sqrt{\pi n}}
            \left[1 + \frac{x^2}{n}\right]^{-(n+1)/2},
            \qquad \qquad -\infty < x < \infty.    \eqlabel{eq:fstudent}
   \endeq
   Gives at least 12 decimals of precision for $n \le 10^3$, and at least
   10 decimals  for $10^3 < n \le 10^5$.
   Restriction:  $n>0$.
  \endtab
\code


double fdist_Student2 (long n, int d, double x);
\endcode
  \tab
  Returns an approximation of the {\em Student-$t$\/} distribution 
  function with $n$ degrees of freedom, with density (\ref{eq:fstudent}).
  Uses the relationship (see \cite{tJOH95a})
 \eq
  2 F(x) = \left\{ \begin{array}{ll}
          I_{n/2, 1/2}(n/(n+x^2))   & \mbox{ for } x<0,\\[5pt]
          I_{1/2, n/2}(x^2/(n+x^2)) & \mbox{ for } x\ge 0,
     \end{array}\right.                       \eqlabel{eq:student-beta}
 \endeq
  where $I_{p,q}$ is the {\em beta\/} distribution function with
  parameters $p$ and $q$
\hrichard {Dans Kotz et Johnson et aussi dans Kendall,
 la distribution {\em beta\/} est
 not\'ee  $I_{p, q}(x)$. Je crois que \c ca pourrait \^etre assez standard. }
  (also called the incomplete {\em beta\/} ratio) defined in 
  (\ref{eq:Fbeta}), which is approximated by calling {\tt fdist\_Beta}. 
  The function tries to return $d$ decimals digits of precision
 (but there is no guarantee).
  Restrictions:  $n>0$ and $0 < d \le 15$.
  \endtab
\code


double fdist_Gamma (double a, int d, double x);
\endcode
  \tab
  Returns an approximation, based on \cite{tBAT70a}, of the {\em gamma\/}
  distribution function with  parameter $a$, whose density is
  \eq
    f(x) = \frac {x^{a-1} e^{-x}}{\Gamma(a)},
  \endeq
  for $x\ge 0$, where $\Gamma$ is the gamma function, defined by
  \eq
    \Gamma(\alpha) = \int_0^\infty x^{\alpha-1} e^{-x} dx.
                                                       \label{eq:Gamma}
  \endeq
%%   One has $\Gamma(\alpha) = (\alpha-1)!$ when $\alpha$ is an integer.
  The function tries to return $d$ decimals digits of precision.
%, but there is no guarantee.  
  For $a$ not too large (e.g., $a \le 1000$),
  $d$ gives a good idea of the precision attained.
%% For d = 16, gives at least 13 decimals of precision for $a \le 1000$,
%% and at least 10 decimals  for $1000 < a \le 100000$.
  For $a \ge 100000$, uses a normal approximation given in \cite{tPEI68a}.
   Restrictions:  $a>0$ and  $0 < d \le 15$.
  \endtab
\code


double fdist_Beta (double p, double q, int d, double x);
\endcode
  \tab
  Returns an approximation of 
  \eq
    F(x) = I_{p,q}(x) 
         = \int_0^x \frac {t^{p-1} (1 - t)^{q-1}}{B(p, q)} dt,
                                                     \eqlabel{eq:Fbeta}
  \endeq
  the {\em beta\/} distribution function with parameters $p$ and $q$, 
  evaluated at $x \in [0, 1]$, where $B(p,q)$ is the {\em beta\/} function
  defined by
  \eq
    B(p,q) = \frac{\Gamma (p) \Gamma (q)}{ \Gamma (p+q)},
  \endeq
  where $\Gamma (x)$ is the Gamma function defined in (\ref{eq:Gamma}).
  For $\max(p, q) \le 1000$, use a recurrence relation in $p$ and $q$ for
  {\tt fdist\_Beta}, given in \cite{tGAU64a,tGAU64b}. 
  Else, if $\min(p, q) \le 30$,
  use an approximation due to Bol'shev \cite{tMAR78a}.  Otherwise, use a
  normal approximation \cite{tPEI68a}.
  The function tries to return $d$ decimals digits of precision.
  For $d\le 13$, when the normal approximation is {\em not\/} used,
  $d$ gives a good idea of the precision attained.
  Restrictions:  $p>0$,  $q>0$, $x \in [0, 1]$ and $0 < d \le 15$.
  \endtab
\code


double fdist_BetaSymmetric (double p, double x);
\endcode
  \tab
  Returns an approximation of the symmetrical {\em beta\/} distribution 
  function $F(x)$ with parameters $p = q$  as defined in (\ref{eq:Fbeta}).
  Uses four different hypergeometric series 
  (for the four cases $x$ close to 0 and $p \le 1$,
     $x$ close to 0 and $p > 1$,  $x$ close to 1/2 and $p \le 1$,
  and  $x$ close to 1/2 and $p > 1$)
  to compute $F(x)$.
  For $p > 100000$, uses a normal approximation given in \cite{tPEI68a}.
  Restrictions:  $p>0$ and $x \in [0, 1]$.
  \endtab
\code


double fdist_KSPlus (long n, double x);
\endcode
 \tab  Returns $p = P[D_n^+ \le x]$, where 
 \eq                                            \eqlabel{eq:KSPlus}
   D_n^+ = \sup_{-\infty < s < \infty} [\hat F_n(s) - F(s)]^+
 \endeq
  is the positive Kolmogorov-Smirnov statistic for a sample of size $n$
  whose empirical distribution function is $\hat F_n$,  
  under the hypothesis that the observations follow a continuous distribution
  function $F$. 
  (Recall that $x^+$ represents $\max (0, x)$, the positive part of $x$.)
  The statistic 
 \eq                                            \eqlabel{eq:KSMinus}
   D_n^- = \sup_{-\infty < s < \infty} [F(s) - \hat F_n(s)]^+
 \endeq
  has the same distribution as $D_n^+$.
  Procedures for computing these statistics are availables in 
  module {\tt gofs}.
  The distribution function of $D_n^+$ can be approximated via the
  following expressions:
  \begin {eqnarray}
   P[D_n^+ \le x]
    &=& 1 - x \sum_{i=0}^{\lfloor n(1-x)\rfloor}
        \left(\matrix{n\cr i\cr}\right)
        \left({i\over n} + x \right)^{i-1}
        \left(1 - {i\over n} - x \right)^{n-i}     \label {DistKS1} \\[4pt]
    &=& x \sum_{j=0}^{\lfloor nx \rfloor}
        \left(\matrix{n\cr j\cr}\right)
        \left({j\over n} - x \right)^j
        \left(1 - {j\over n} + x \right)^{n-j-1}   \label {DistKS2} \\[4pt]
    &\approx& 1 - e^{-2 n x^2} \left[1 - {2x\over 3} \left(
           1 - x\left(1 - {2 n x^2 \over 3}\right) \right.\right. 
                                                   \nonumber\\[4pt]
    &&  \left.\left. - {2\over 3n} \left({1\over 5} - {19 n x^2 \over 15}
              + {2n^2 x^4 \over 3}\right) \right) + O(n^{-2}) \right].
                                                   \label {DistKS3}
  \end {eqnarray}
  Formula (\ref{DistKS1}) and (\ref{DistKS2}) can be found in 
  \cite{tDUR73a}, equations (2.1.12) and (2.1.16), while (\ref{DistKS3})
  can be found in \cite{tDAR60a}.
  Formula (\ref{DistKS2}) contains less terms than (\ref{DistKS1})
  when $x < 0.5$, but becomes numerically unstable as $nx$ increases.
%  because its terms  alternate in sign and become large
%  (in absolute value) compared to their sum.
  The approximation (\ref{DistKS3}) is simpler to compute and excellent
  when  $nx$ is large.
  Our implementation uses (\ref{DistKS2}) when $nx < 6.5$,
  (\ref{DistKS1}) when $nx \ge 6.5$ and $n \le 4000$,
  and (\ref{DistKS3}) when $nx \ge 6.5$ and $n > 4000$.
%   The relative  error on $p(x) = P[D_n^+ \le x]$ is always less than
%   $10^{-5}$, and the relative error on $1-p(x)$ is less than 
%   $10^{-1}$ when $1-p(x) > 10^{-10}$.
%   The {\em absolute\/}  error on $1-p(x)$ is less than $10^{-11}$ 
%   when $1-p(x) < 10^{-10}$.
%  It must be noted that Korolyuk \cite{tKOR59a} and many others 
%  gave an erroneous formula for the asymptotic form (\ref{DistKS3}).
%  Darling \cite{tDAR60a} gives the correct formula with references.
  \endtab
\code


double fdist_KS1 (long n, double x);
\endcode
 \tab Returns $u = P[D_n \le x]$ % using the program described in \cite{LECz09},
   where $D_n = \max\{D_n^+, D_n^-\}$
  is the two-sided Kolmogorov-Smirnov statistic \cite{tBRO07a} for a sample
  of size $n$, and $D_n^+$ and $D_n^-$ are defined in (\ref{eq:KSPlus}) and
  (\ref{eq:KSMinus}).
 This method uses Pomeranz's recursion formula \cite{tBRO08a,tPOM74a} for
$n \le 400$, which return at least 13 decimal digits of precision. It uses
the Pelz-Good asymptotic expansion \cite{tPEL76a} in the central part of
 the range for $n > 400$ and returns at least 6 decimal digits of precision
 everywhere for $400 < n \le 4000$. For $n > 4000$, it returns at least
 2 decimal digits of precision for all $u > 10^{-22}$, and at least
 5 decimal digits of precision for all $u > 10^{-7}$.
For a given $n$, the precision increases as $x$ increases.
 This method is much faster than  {\tt fdist\_KS2} for moderate or large $n$.
 \endtab
\code


double fdist_KS2 (long n, double x);
\endcode
\tab Another version of the Kolmogorov-Smirnov distribution
$ P[D_n \le x]$, using Durbin's matrix formula \cite{tDUR73a}. 
It is astronomically slow for large $n$. According to its authors 
\cite{tMAR03a}, it should return at least 7 decimal digits
of precision.
\endtab
\code


double fdist_KSPlusJumpOne (long n, double a, double x);
\endcode
 \tab
  Similar to {\tt fdist\_KSPlus} but for the case where the distribution
  function $F$ has a jump of size $a$ at a given point $x_0$,
  is zero at the left of $x_0$,
  and is continuous at the right of $x_0$.
  The Kolmogorov-Smirnov statistic is defined in that case as
  \eq
    D_n^+(a) = \sup_{a\le u\le 1} \left(\hat F_n(F^{-1}(u)) - u\right)
             = \max_{\rule{0pt}{7pt} \lfloor 1+an\rfloor \le j \le n}
               \left(j/n - F(V_{(j)})\right).
                                    \eqlabel{eq:KSPlusJumpOne}
 \endeq
\iffalse  %%%%%
  and 
  \eq
    D_n^-(a) = \sup_{a\le u\le 1} \left(u - \hat F_n(F^{-1}(u))\right)
             = \max_{\rule{0pt}{7pt} \lfloor 1+an\rfloor \le j \le n}
               \left(F(V_{(j)})-(j-1)/n\right),
  \endeq
 \pierre {It seems that $D_n^-(a)$ has a {\em different\/} distribution
    function. }
\fi  %%%%
  where $V_{(1)},\dots,V_{(n)}$ are the observations sorted by increasing
  order.  The procedure returns an approximation of 
  $P[D_n^+(a) \le x]$ computed via
  \begin {eqnarray}
   P[D_n^+(a) \le x]
    &=& 1 - x \sum_{i=0}^{\lfloor n(1-a-x)\rfloor}
        \left(\matrix{n\cr i\cr}\right)
        \left({i\over n} + x \right)^{i-1}
        \left(1 - {i\over n} - x \right)^{n-i}   \eqlabel{DistKSJ1} \\[6pt]
    &=& x \sum_{j=0}^{\lfloor n(a+x) \rfloor}
        \left(\matrix{n\cr j\cr}\right)
        \left({j\over n} - x \right)^j
        \left(1 - {j\over n} + x \right)^{n-j-1}.  \eqlabel{DistKSJ2}
  \end {eqnarray}
%  La fonction de r\'epartition est la m\^eme pour  --->  WRONG!
%  \eq
%    D_n^-(a) = \sup_{a\le u\le 1} (u - \hat F(F^{-1}(u)))
%             = \max_{\lfloor 1+an\rfloor \le j \le n} (F(T_{(j)})-(j-1)/n),
%  \endeq
%  et aussi lorsque le supremum est sur $0\le u \le 1-a$ au lieu de
%  $a\le u\le 1$.
 \hpierre {R\'ef\'erences pour ces formules?}
  The current implementation  uses  formula (\ref{DistKSJ2})
  when $n(x+a) < 6.5$ and $x+a < 0.5$, and uses  (\ref{DistKSJ1})
  when $nx \ge 6.5$ or $x+a \ge 0.5$.
  Restriction: $0 < a < 1$.
  \endtab
\hide  %%%%%%%%%%%%%%%%%%%%%%%%%%%%%%%%  
\code
#if 0

void fdist_FindJumps (fdist_FUNC_JUMPS *H, int Detail);
\endcode
 \tab
   Find empirically all the jumps of the function  {\tt H->F(H->par, x)}
   in the interval  {\tt (H->xa, H->xb)} by advancing with steps of size 
   {\tt epsX} along the  {\tt x} direction. If the value {\tt y} of the
   function differs by more than  {\tt epsY} over such a step, it will be
   considered as a possible jump (discontinuity) of the function.
   If {\tt epsX} is chosen very small, the time taken to explore the
   interval could be very long. If {\tt epsX} is chosen very large, some
   jumps will not be found.
   {\tt epsY/epsX} should be chosen larger than the largest
   slope of the function at continuous points in the interval.
   The number of jumps is returned in  {\tt H->NJumps}; the position
   {\tt x} of the $i$-th jump will be kept in 
   {\tt H->xJump[i], i = 1,2,...,H->NJumps}.
   The value of the function  immediately to the left and to the right
   of jump $i$ will be kept in {\tt H->yLeftJump[i]} and 
   {\tt H->yRightJump[i]} respectively. The name of the function can be
   kept in {\tt H->doc}.
   If {\tt Detail $>$ 0}, very detailed information will be printed about
   the jumps.
 \endtab
\code


void fdist_FreeJumps (fdist_FUNC_JUMPS *H);
\endcode
 \tab
  Free the memory allocated internally by {\tt fdist\_FindJumps} (and only
  that memory, that is {\tt xJump, yLeftJump, yRightJump})
  to keep information about the jumps of {\tt H}.
  This should be called when one has no more use for the jumps. {\tt H}
  itself and the other fields of {\tt H} that are allocated by
  the user  (v.g. {\tt par, doc}) must be released explicitly by the user. 
 \endtab
\code


double fdist_KSMinusJumpsMany (fdist_FUNC_JUMPS *H, double x);
\endcode
 \tab  
  Return the $p$-value $P[D_{\!N}^- \ge x]$, where $D_{\!N}^-$ is the
  Kolmogorov-Smirnov statistic (\ref{eq:KSMinus}) for a sample of size
  $N =$  {\tt H->par->paramI[0]}, under
  the hypothesis that the observations follow a  law with a discontinuous
  cumulative distribution function  {\tt H->F}. The distribution function
  {\tt H->F} has {\tt H->NJumps} jumps;  the $x$ coordinate of jump $i$
  is in {\tt H->xJump[i], i = 1, 2, ..., NJumps}, and the values of {\tt H->F}
  immediately to the left and to the right of jump $i$ are in 
  {\tt H->yLeftJump[i]} and {\tt H->yRightJump[i]}.

  This distribution function is computed following the method
  proposed in \cite{tCON72a}. The sample size  $N$ should not exceed
  64, otherwise loss of precision in the computations
  will make the results meaningless.
  \endtab
\code


double fdist_KSPlusJumpsMany (fdist_FUNC_JUMPS *H, double x);
#endif
\endcode
 \tab  
  Similar to {\tt fdist\_KSMinusJumpsMany}, but for $D_N^+$, defined
  in (\ref{eq:KSPlus}).
  \endtab
\endhide  %%%%%%%%%%%%%%%%%%%%%
\code


double fdist_CramerMises (long n, double x);
\endcode
 \tab  Returns an approximation of $P[W_n^2 \le x]$, where $W_n^2$ is the
  Cram\'er von Mises  statistic (see \cite{tSTE70a,tSTE86b,tAND52a,tKNO74a})
  defined in (\ref{eq:CraMis}),
  for a sample of independent uniforms over $(0,1)$.
  The approximation is based on the
  distribution function of $W^2 = \lim_{n\to\infty} W_n^2$, which has
  the following series expansion derived
  by Anderson and Darling \cite{tAND52a}:
   \begin{equation}
   P(W^2 \le x)  \ = \ \frac1{\pi\sqrt x} \sum_{j=0}^\infty (-1)^j {-1/2
   \choose j} \sqrt{4j+1}\;\; {\rm exp}\left\{-\frac{(4j+1)^2}{16 x}\right\}
    {\rm K}_{1/4}\left(\frac{(4j+1)^2}{16 x}\right),
                                                       \eqlabel{eq:DistCM1}
   \end{equation}
  where ${\rm K}_{\nu}$ is the  modified Bessel function of the 
  second kind.
  To correct for the deviation between $P(W_n^2\le x)$ and $P(W^2\le x)$,
  we add a correction in $1/n$, obtained empirically by simulation.
  For $n = 10$, 20, 40, the error is less than
  0.002, 0.001, and 0.0005, respectively, while for
  $n \ge 100$ it is less than 0.0005.
  For $n \to\infty$, we estimate that the procedure returns
  at least 6 decimal digits of precision.
  For $n = 1$, the procedure computes the exact distribution:
  $P(W_1^2 \le x) = 2 \sqrt {x - 1/12}$ for $1/12 \le x \le 1/3$.
 \endtab
\code


double fdist_WatsonG (long n, double x);
\endcode
 \tab Returns an approximation of $P[G_n \le x]$, where $G_n$ is the
  Watson statistic  defined in (\ref{eq:WatsonG}),
  for a sample of independent uniforms over $(0,1)$.
  The approximation is computed in a similar way as for 
  {\tt fdist\_CramerMises}.
  To implement this procedure, a table of the values of
  $g(x) = \lim_{n\to\infty} P[G_n \le x]$ and of its derivative 
  was first computed by numerical integration. 
  For $x \le 1.5$, the procedure uses this table with cubic spline 
  interpolation.
  For $x > 1.5$, it uses the empirical curve $g(x) = 1 - e^{19 - 20x}$.
  A correction of order $1/\sqrt{n}$, obtained
  empirically from $10^7$ simulation runs with $n = 256$ and also
  implemented as an interpolation table with an exponential tail,
  is then added.
  The  absolute  error is estimated to be less than 
  0.01, 0.005, 0.002, 0.0008, 0.0005, 0.0005, 0.0005 for 
  $n = 16$, 32, 64, 128, 256, 512, 1024, respectively.
  For the trivial case $n=1$, always returns 0.5.
 \endtab
\code


double fdist_WatsonU (long n, double x);
\endcode
 \tab  Returns $P[U^2 \le x]$, where $U^2$ is the Watson statistic 
  defined in (\ref{eq:WatsonU}) in the limit when $n \to\infty$,
  for a sample of independent uniforms over $(0,1)$.
  Only this limiting distribution (when $n \to\infty$) is implemented. 
  It is given by
   \begin{equation}
    P(U^2 \le x)  \ = \ 1 + 2 \sum_{j=1}^\infty (-1)^j e^{-2 j^2 \pi^2 x}
                                                      \eqlabel{eq:DistWU1}
   \end{equation}
  This sum converges extremely fast except for small $x$, where alternating
  successive terms give rise to numerical instability.
  But with the Poisson summation formula \cite{mLAN73a},
  the sum can be transformed to
  \begin{equation}
     P(U^2 \le x)  \ = \  \sqrt{\frac2{\pi x}}\; \sum_{j=0}^\infty
        e^{- (2 j+1)^2/8 x}                           \eqlabel{eq:DistWU2}
  \end{equation}
  which can be used for small $x$.
  The current implementation uses (\ref{eq:DistWU1}) for $x > 0.15$,
  and (\ref{eq:DistWU2}) for $x \le 0.15$.
  The absolute difference between the returned value and $P[U_n^2 \le x]$ 
  is estimated to be less than 0.01 for $n \ge 8$. 
  For the trivial case $n=1$, always returns 0.5.
 \endtab
\code


double fdist_AndersonDarling (long n, double x);
\endcode
 \tab Returns $F_n(x) = P[A_n^2 \le x]$, where $A_n^2$ is the
  Anderson-Darling statistic \cite{tAND52a} defined in (\ref{eq:Andar}),
  for a sample of independent uniforms over $(0,1)$.
  The approximation is computed similarly as for 
  {\tt fdist\_CramerMises}.
  To implement this procedure, an interpolation table of the values of
  $F(x) = \lim_{n\to\infty} P[A_n^2 \le x]$
  was first computed by numerical integration. 
  Then a linear correction in $1/n$, obtained by simulation, was added.
  For $x \le 5$, the procedure approximates $F_n(x) = P[A_n^2 \le x]$ by
  interpolation.  For $5 < x < 10$, it uses the empirical curve
  $F_n(x) \approx 1 - e^{-1.06x - 0.56} - e^{-1.06x - 1.03}/n$,
  which includes the empirical  correction in $1/n$.
  The absolute error on $F_n(x)$ is estimated to be
  less than $0.001$ for $n > 6$ except far in the tails.
  For $n = 2$, 3, 4, 6, it is estimated to be
  less than  0.04, 0.01, 0.005, 0.002, respectively.
  In the lower tail ($x < 0.2$), the approximation (3.6) of Sinclair
   and Spurr \cite{tSIN88a}
$$
  F(x) = 1 - \frac1{1 + \exp{\left(1.784 + 0.9936x +
 \frac{0.03287}{x}
    - \frac{(2.018 + {0.2029}/{x})}{\sqrt x}\right)}}
$$
  is used without correction for finite $n$. In the far upper tail
  ($x > 10$), the approximation (3.5) of Sinclair
   and Spurr \cite{tSIN88a}
$$
      F(x) =  1 - \frac{1.732  \exp(-x)}{\sqrt{\pi x}}
$$
 is used without correction for finite $n$.
  For $n=1$, the procedure returns the exact value,
  $F_1(x) = \sqrt{1 - 4e^{-x-1}}$ for $x\ge \ln(4) - 1$.
 \endtab
\code


double fdist_AndersonDarling2 (long n, double x);
\endcode
 \tab Returns the value of the Anderson-Darling distribution at $x$
  for a sample of $n$ independent uniforms over $(0,1)$ using Marsaglia's
  and al. algorithm \cite{tMAR04a}. First the limiting distribution
  for $n\to\infty$ is computed to within 6-digit accuracy according to the
  authors. Then an empirical correction obtained by simulation is added
  for finite $n$.
  For $n=1$, the procedure returns the exact value,
  $F_1(x) = \sqrt{1 - 4e^{-x-1}}$ for $x\ge \ln(4) - 1$.
 \endtab



%%%%%%%%%%%%%%%%%%%%%%%%%%%%%%%%%%%%%%%%%%
\guisec{Discrete distributions}

\code

double fdist_Geometric (double p, long s);
\endcode
  \tab Returns
  \eq
   F(s) = \sum_{j = 0}^s p\, (1-p)^{j} ~=~ 1 - (1-p)^{s+1},
                                               \eqlabel{eq:Fgeometric}
  \endeq
  the distribution function of a geometric random variable with
  parameter $p$, evaluated at $s$.
  Restriction: $0 \le p \le 1$.
 \endtab
\code


double fdist_Poisson1 (double lambda, long s);
\endcode
  \tab  Returns
  \eq
    F_\lambda(s) =  e^{-\lambda} \sum_{j=0}^s\; \frac{\lambda^j}{j!},
                                               \eqlabel{eq:FPoisson}
  \endeq
  the Poisson distribution function 
  with parameter $\lambda =$ {\tt lambda}, evaluated at $s$.
\ifdetailed
  This procedure adds all the non-negligible terms of the sum
  if $\lambda \le 150$; otherwise it uses the relationship
  $F_\lambda(s) = 1 - G_{s + 1}(\lambda)$, in which $G_{s+1}$ is the
  gamma distribution with parameter $s+1$, computed via {\tt
  fdist\_Gamma}.
\fi
 \hpierre{Suggestions: lorsque $s>\lambda$, calculer plutot
   les termes de $1-F(s)$ ? Et si $\lambda > 150$ mais $s$ est petit,
   ne vaut-il pas mieux utiliser la somme? Peut-etre mettre la limite
   sur $s$ a la place? }
 \hrichard{ Si $\lambda$ est grand, on risque d'avoir d\'epassement de
   capacit\'e avec le facteur $e^{-\lambda}$ et l'autre facteur aussi,
   si on somme les termes
   explicitement. D'ailleurs, avec grand $\lambda$, la probabilit\'e
   d'avoir des petits $s$ est n\'egligeable. C'est pourquoi j'ai mis la
   limite sur $\lambda$ et non sur $s$.\\ Pour $s>\lambda$, la queue
   de Poisson \'etant plus allong\'ee vers le haut que vers le bas, il
   n'est pas \'evident que ce serait plus efficace de calculer
   $1-F(s)$,  et ce serait beaucoup plus compliqu\'e.}
 \hpierre {Et pourquoi prendre $10^5$ au lieu de 150 dans CreatePoisson?
   Si on a 5 a 10 appels a faire, il me semble qu'appeler ``gamma''
   est beaucoup plus efficace que de calculer des tableaux de taille
   $10^5$? }
 \hrichard {Dans le cas o\`u $\lambda = 10^5$, on ne calcule pas des
   tableaux de taille $10^5$, mais 2 tableaux de taille 4849. Les
   autres termes sont $< 10^{-16}$ et ne sont ni calcul\'es ni
   conserv\'es.}
  In the cases where the Poisson distribution must be computed more than
  once with the same $\lambda$, % and $\lambda$ is not too large,
  it is more efficient to use 
  {\tt fdist\_Poisson2} instead of {\tt fdist\_Poisson1}.
  Restriction: $\lambda > 0$.
 \endtab
\code


double fdist_Poisson2 (fmass_INFO W, long s);
\endcode
 \tab  Returns the Poisson distribution function (\ref{eq:FPoisson})
  from the structure {\tt W}, which must have been created previously
  by calling {\tt fmass\_CreatePoisson} with the desired $\lambda$.
%  Restriction: only integral values of $s$ are meaningful. 
 \endtab
\code


double fdist_Binomial1 (long n, double p, long s);
\endcode
  \tab  Returns
  \eq
    F(s) = \sum_{j=0}^s {n \choose j}\; p^j (1-p)^{n-j},
                                               \eqlabel{eq:Fbinomial}
  \endeq
  the distribution function of a binomial random variable with parameters
  $n$ and $p$, evaluated at $s$.
\ifdetailed
  If $n \le 10000$, the non-negligible terms of the sum are 
  added explicitly.
  If $n > 10000$ and $np(1-p) > 100$, the Camp-Paulson normal 
  approximation \cite{tCAM51a,tMOL70a} is used, otherwise, a Poisson
  approximation due to Bol'shev \cite{tBOL64a,tMOL70a} is used.
\fi
  When the binomial distribution has to be computed more than
  once with the same parameters $n$ and $p$, it is more efficient to use 
  {\tt fdist\_Binomial2} instead of {\tt fdist\_Binomial1}, 
  unless $n$ is very large (e.g., $n > 10^5$).
  Restrictions: $0 \le p \le 1$ and $n \ge 0$.
 \endtab
\code


double fdist_Binomial2 (fmass_INFO W, long s);
\endcode
 \tab  Returns the binomial distribution function (\ref{eq:Fbinomial})
  from the structure {\tt W}, which must have been created previously
  by calling {\tt fmass\_CreateBinomial} with the desired values of
  $n$ and $p$.
 \endtab
\code


double fdist_NegaBin1 (long n, double p, long s);
\endcode
  \tab Returns
  \eq
   F(s) = \sum_{j = 0}^s {n + j - 1 \choose j} p^n (1-p)^{j},
                                               \eqlabel{eq:Fnegabin}
  \endeq
  the distribution function of a negative binomial random variable with
  parameters $n$ and $p$, evaluated at $s$. 
  If this distribution has to be computed more than
  once with the same $n$ and $p$, it is more efficient to use 
  {\tt fdist\_NegaBin2} instead of {\tt fdist\_NegaBin1},
  unless $n$ is very large. 
% (e.g., $n > 10000$). ??
  Restrictions: $n \ge 0$ and $0 \le p \le 1$.
 \endtab
\code


double fdist_NegaBin2 (fmass_INFO W, long s);
\endcode
 \tab  Returns the negative binomial distribution function 
  (\ref{eq:Fnegabin}) from the structure {\tt W}, 
  which must have been created previously
  by calling {\tt fmass\_CreateBinomial} with the desired values of
  $n$ and $p$.
 \endtab
\code


double fdist_Scan (long N, double d, long m);
\endcode
 \tab Returns $F(m)$, the distribution function of the scan statistic
  with parameters $N$ and $d$, evaluated at $m$.
  For a description of this statistic and its distribution, see 
  {\tt fbar\_Scan}, which computes its complementary distribution
  $\bar F(m) = 1 - F(m-1)$.
 \endtab
\code
\hide
#endif
\endhide
\endcode
