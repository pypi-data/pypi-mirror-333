\defmodule {fspectral}

This module applies spectral tests from the module {\tt sspectral}
to a family of generators of different sizes.

\bigskip
\hrule
\code\hide
/* fspectral.h  for ANSI C */
#ifndef FSPECTRAL_H
#define FSPECTRAL_H
\endhide
#include "ffam.h"
#include "fres.h"
#include "fcho.h"


extern long fspectral_MaxN;
\endcode
\tab
  Upper bound on $N$.
  When $N$ exceeds its limit value, the tests are not done.
  Default value: $N = 2^{22}$.
\endtab



%%%%%%%%%%%%%%%%%%%%%%%%%%%%%%%%%%%%%%%%%%
\guisec{The tests}

\code
void fspectral_Fourier3 (ffam_Fam *fam, fres_Cont *res, fcho_Cho *cho,
                         int k, int r, int s,
                         int Nr, int j1, int j2, int jstep);
\endcode
\tab
 This function calls the test {\tt sspectral\_Fourier3} with parameters
 $N$, {\tt k}, {\tt r}, and {\tt s} for sample size $N$ chosen by
  $N = {}$ {\tt cho->Choose(param, i, j)},
 for the first {\tt Nr} generators of family {\tt fam}, for $j$ going from
 {\tt j1} to {\tt j2} by steps of {\tt jstep}. The parameters in {\tt param}
 were set at the creation of {\tt cho} and $i$ is the lsize of the
 generator being tested.
 When $N$ exceeds {\tt fspectral\_MaxN}, the test is not done.
\endtab

\code
\hide
#endif
\endhide
\endcode
