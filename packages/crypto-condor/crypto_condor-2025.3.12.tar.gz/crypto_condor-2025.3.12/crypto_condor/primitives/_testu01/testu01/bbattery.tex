\defmodule {bbattery}

This module contains predefined batteries of statistical tests for
sources of random bits or sequences of uniform random numbers in
the interval $[0, 1)$.
\hpierre{The choice of tests and parameters in these batteries is
        subject to change in the near future,
        following further experimentations.}
To test a RNG for general use, one could first apply the
small and fast battery {\tt SmallCrush}.
If it passes, one could then apply the more stringent battery
{\tt Crush}, and finally the yet more time-consuming
battery {\tt BigCrush}.
The batteries {\tt Alphabit} and {\tt Rabbit} can be applied on a binary
file considered as a source of random bits. They can also be applied
on a programmed generator.  {\tt Alphabit} has been defined primarily to test
{\it hardware} random bits generators.
The battery {PseudoDIEHARD} applies most of the tests in
the well-known {\it DIEHARD\/} suite of Marsaglia \cite{rMAR96b}.
The battery {\tt FIPS\_140\_2} implements the small suite of tests
of the {\it FIPS-140-2} standard from NIST.

The batteries described in this module  will write the results of each test
(on standard output) with a standard level of details (assuming that the
 boolean switches of module {\tt swrite} have their default values),
followed by a summary  report of the suspect $p$-values obtained from the
 specific tests included in the batteries.
It is also possible to get only the summary report in the output,
with no detailed output from the tests,
by setting the boolean switch {\tt swrite\_Basic} to {\tt FALSE}.

Some of the tests compute more than one statistic using the same stream of
random numbers and these statistics are thus not independent.
\emph{That is why the number of statistics in the summary reports is larger than
the number of tests in the description of the batteries.}

%%%%%%%%%%%%%%%%%%%%%%%%%%%%%%%%%%%%%%%%%%%%%%%%%%%%%%%%%%%%%%
\bigskip
\hrule
\code\hide
/* bbattery.h for ANSI C */
#ifndef BBATTERY_H
#define BBATTERY_H
\endhide
#include "unif01.h"


extern int bbattery_NTests;
\endcode
  \tab The maximum number of $p$-values in the array {\tt bbattery\_pVal}.
  For small sample size, some of the tests in the battery may not be done.
  Furthermore, some of the tests computes more than one statistic and
  its $p$-value, so {\tt bbattery\_NTests} will usually be larger than
  the number of tests in the battery.
  \endtab
\code


extern double bbattery_pVal[];
\endcode
  \tab This array keeps the $p$-values resulting from the  battery of tests
  that is currently applied (or the last one that has been called). It is
  used by any battery in this module. The $p$-value of the $j$-th test in
  the battery is kept in {\tt bbattery\_pVal[$j-1$]}, for $1\le j\le $
  {\tt bbattery\_NTests}.
%% When a particular test in the battery
%%   is not done (because the sample size is too small), then the
%%  corresponding $p$-value is set to $-1$.
 \endtab
\code


extern char *bbattery_TestNames[];
\endcode
  \tab This array keeps the names of each test from  the
  battery that is currently applied (or the last one that has
  been called). It is used by any battery in this module.
  The name of the $j$-th test in the battery is
   kept in {\tt bbattery\_TestNames[$j-1$]},
  for $1\le j\le $ {\tt bbattery\_NTests}.
 \endtab


\guisec{The batteries of tests}

\code

void bbattery_SmallCrush (unif01_Gen *gen);

void bbattery_SmallCrushFile (char *filename);
\endcode
  \tab  Both functions applies {\tt SmallCrush}, a small and fast battery
   of tests, to a RNG. The function {\tt bbattery\_Small\-CrushFile} applies
  {\tt SmallCrush} to a RNG given as a text file of floating-point numbers in
  $[0, 1)$; the file requires slightly less than 51320000 random numbers.
   \label{bat:SmallCrush}
   The file will be rewound to the beginning before each test.
  \index{SmallCrush}\index{battery of tests!SmallCrush}%
  Thus  {\tt bbattery\_SmallCrush} applies the tests on one unbroken
  stream of successive numbers, while {\tt bbattery\_Small\-CrushFile}
  applies each test on the same sequence of numbers.
  Some of these tests assume that {\tt gen} returns at least 30 bits of
  resolution; if this is not the case, then the generator is most likely to
  fail these particular tests.

  The following tests are applied:
  \endtab
  \begin{enumerate}
  \item {\tt smarsa\_BirthdaySpacings} with $N=1$,  $n=5*10^6$,  $r=0$,
   $d = 2^{30}$, $t=2$, $p=1$.

  \item {\tt sknuth\_Collision} with $N=1$,  $n=5*10^6$,  $r=0$,
   $d = 2^{16}$, $t=2$.

  \item {\tt sknuth\_Gap}  with $N=1$,  $n=2*10^5$,  $r=22$, {\tt Alpha} $=0$,
    {\tt Beta} $=1/256$.

  \item {\tt sknuth\_SimpPoker}  with $N=1$,  $n=4*10^5$,  $r=24$,
   $d = 64$, $k=64$.

  \item {\tt sknuth\_CouponCollector}  with $N=1$, $n=5*10^5$, $r=26$,
   $d=16$.

  \item {\tt  sknuth\_MaxOft}  with $N=1$,  $n=2*10^6$,  $r=0$,
   $d = 10^5$, $t=6$.

  \item {\tt  svaria\_WeightDistrib} with $N=1$, $n=2*10^5$, $r=27$,
   $k=256$, {\tt Alpha} $=0$,  {\tt Beta} $=1/8$.

  \item {\tt smarsa\_MatrixRank} with $N=1$,  $n=20000$,  $r=20$, $s=10$,
   $L = k=60$.

  \item {\tt sstring\_HammingIndep} with $N=1$, $n=5*10^5$,  $r=20$, $s=10$,
   $L = 300$, $d = 0$.

  \item {\tt swalk\_RandomWalk1} with $N=1$, $n=10^6$,  $r=0$, $s=30$,
   $L_0 = 150$, $L_1=150$.

  \end{enumerate}

\code


void bbattery_RepeatSmallCrush (unif01_Gen *gen, int rep[]);
\endcode
  \tab This function applies specific tests of {\tt SmallCrush} on
   generator {\tt gen}. Test numbered $i$ in the enumeration above will
  be applied {\tt rep[$i$]} times successively on  {\tt gen}. Those tests with
   {\tt rep[$i$]} = 0 will not be applied. This is useful when a test in
    {\tt SmallCrush} had a suspect $p$-value, and one wants to reapply the
  test a few more times to find out whether the generator failed
  the test or whether the suspect $p$-value was a statistical fluke.
  Restriction: Array {\tt rep} must have one more element than the
  number of tests in {\tt SmallCrush}.
 \endtab
\bigskip

\hrule
\code


void bbattery_Crush (unif01_Gen *gen);
\endcode
  \tab
  Applies the battery {\tt Crush}, a suite of stringent statistical
  tests, to the generator {\tt gen}.
  \index{Crush}\index{battery of tests!Crush}%
  The battery includes the classical tests described in Knuth \cite{rKNU98a}
  as well as many other tests. Some of these tests assume that
  {\tt gen} returns at least 30 bits of resolution; if that
  is not the case, then the generator will certainly
  fail these particular tests. One test requires 31 bits of resolution:
  the {\tt BirthdaySpacings} test with $t=2$.
  On a PC with an AMD Athlon 64 Processor 4000+
  of clock speed 2400 MHz running with Red Hat Linux, Crush
  will require around 1 hour of CPU time. {\tt Crush} uses approximately
  $2^{35}$ random numbers.
%%  Some of the tests will require as much as 350 Megs of RAM memory.
  The following tests are applied: \label{bat:Crush}
  \endtab

  \begin{enumerate}
  \item {\tt smarsa\_SerialOver} with $N=1$,  $n=5*10^8$,  $r=0$,
  $d = 2^{12}$, $t=2$.

  \item {\tt smarsa\_SerialOver} with $N=1$,  $n=3*10^8$,  $r=0$,
  $d = 2^6$, $t=4$.

  \item {\tt smarsa\_CollisionOver} with $N=10$,  $n=10^7$,  $r=0$,
  $d = 2^{20}$, $t=2$.

  \item {\tt smarsa\_CollisionOver} with $N=10$,  $n=10^7$,  $r=10$,
  $d = 2^{20}$,  $t=2$.

  \item {\tt smarsa\_CollisionOver} with $N=10$,  $n=10^7$,  $r=0$,
  $d = 2^{10}$, $t=4$.

  \item {\tt smarsa\_CollisionOver} with $N=10$,  $n=10^7$,  $r=20$,
  $d = 2^{10}$,  $t=4$.

  \item {\tt smarsa\_CollisionOver} with $N=10$,  $n=10^7$,  $r=0$,
  $d = 32$, $t=8$.

  \item {\tt smarsa\_CollisionOver} with $N=10$,  $n=10^7$,  $r=25$,
  $d = 32$,  $t=8$.

  \item {\tt smarsa\_CollisionOver} with $N=10$,  $n=10^7$,  $r=0$,
  $d = 4$, $t=20$.

  \item {\tt smarsa\_CollisionOver} with $N=10$,  $n=10^7$,  $r=28$,
  $d = 4$,  $t=20$.

  \item {\tt smarsa\_BirthdaySpacings}  with $N=5$,  $n=2*10^7$, $r=0$,
   $d = 2^{31}$, $t=2$, $p=1$.

  \item {\tt smarsa\_BirthdaySpacings}  with $N=5$,  $n=2*10^7$, $r=0$,
   $d = 2^{21}$, $t=3$, $p=1$.

  \item {\tt smarsa\_BirthdaySpacings}  with $N=5$,  $n=2*10^7$, $r=0$,
   $d = 2^{16}$, $t=4$, $p=1$.

  \item {\tt smarsa\_BirthdaySpacings}  with $N=3$,  $n=2*10^7$, $r=0$,
   $d = 2^{9}$, $t=7$, $p=1$.

  \item {\tt smarsa\_BirthdaySpacings}  with $N=3$,  $n=2*10^7$, $r=7$,
   $d = 2^{9}$, $t=7$, $p=1$.

  \item {\tt smarsa\_BirthdaySpacings}  with $N=3$,  $n=2*10^7$, $r=14$,
   $d = 2^{8}$, $t=8$, $p=1$.

  \item {\tt smarsa\_BirthdaySpacings}  with $N=3$,  $n=2*10^7$, $r=22$,
   $d = 2^{8}$, $t=8$, $p=1$.

  \item {\tt snpair\_ClosePairs}  with $N=10$,  $n=2*10^6$, $r=0$, $t=2$,
   $p=0$,  $m=30$.

  \item {\tt snpair\_ClosePairs}  with $N=10$,  $n=2*10^6$, $r=0$, $t=3$,
   $p=0$,  $m=30$.

  \item {\tt snpair\_ClosePairs}  with $N=5$,  $n=2*10^6$, $r=0$, $t=7$,
   $p=0$,  $m=30$.

  \item {\tt snpair\_ClosePairsBitMatch}  with $N=4$,  $n=4*10^6$,
   $r=0$, $t=2$.

  \item {\tt snpair\_ClosePairsBitMatch}  with $N=2$,  $n=4*10^6$,
   $r=0$, $t=4$.

  \item {\tt sknuth\_SimpPoker} with $N=1$,  $n=4*10^7$,  $r=0$, $d=16$, $k=16$.

  \item {\tt sknuth\_SimpPoker} with $N=1$,  $n=4*10^7$,  $r=26$, $d=16$, $k=16$.

  \item {\tt sknuth\_SimpPoker} with $N=1$,  $n=10^7$,  $r=0$, $d=64$, $k=64$.

  \item {\tt sknuth\_SimpPoker} with $N=1$,  $n=10^7$,  $r=24$, $d=64$, $k=64$.

  \item {\tt sknuth\_CouponCollector} with $N=1$, $n=4*10^7$, $r=0$, $d=4$.

  \item {\tt sknuth\_CouponCollector} with $N=1$, $n=4*10^7$, $r=28$, $d=4$.

  \item {\tt sknuth\_CouponCollector} with $N=1$, $n=10^7$, $r=0$, $d=16$.

  \item {\tt sknuth\_CouponCollector} with $N=1$, $n=10^7$, $r=26$, $d=16$.

  \item {\tt sknuth\_Gap} with $N=1$, $n=10^8$, $r=0$, {\tt Alpha} $=0$,
    {\tt Beta} $=1/8$.

  \item {\tt sknuth\_Gap} with $N=1$, $n=10^8$, $r=27$, {\tt Alpha} $=0$,
    {\tt Beta} $=1/8$.

  \item {\tt sknuth\_Gap} with $N=1$, $n=5*10^6$, $r=0$, {\tt Alpha} $=0$,
   {\tt Beta} $=1/256$.

  \item {\tt sknuth\_Gap} with $N=1$, $n=5*10^6$, $r=22$, {\tt Alpha} $=0$,
    {\tt Beta} $=1/256$.

  \item {\tt sknuth\_Run}  with $N=1$, $n=5*10^8$, $r=0$, {\tt Up = TRUE}.

  \item {\tt sknuth\_Run}  with $N=1$, $n=5*10^8$, $r=15$, {\tt Up = FALSE}.

  \item {\tt sknuth\_Permutation} with $N=1$, $n=5*10^7$, $r=0$, $t=10$.

  \item {\tt sknuth\_Permutation} with $N=1$, $n=5*10^7$, $r=15$, $t=10$.

  \item {\tt sknuth\_CollisionPermut} with $N=5$, $n=10^7$, $r=0$, $t=13$.

  \item {\tt sknuth\_CollisionPermut} with $N=5$, $n=10^7$, $r=15$, $t=13$.

  \item {\tt sknuth\_MaxOft} with $N=10$, $n=10^7$, $r=0$, $d=10^5$, $t=5$.

  \item {\tt sknuth\_MaxOft} with $N=5$, $n=10^7$, $r=0$, $d=10^5$, $t=10$.

  \item {\tt sknuth\_MaxOft} with $N=1$, $n=10^7$, $r=0$, $d=10^5$, $t=20$.

  \item {\tt sknuth\_MaxOft} with $N=1$, $n=10^7$, $r=0$, $d=10^5$, $t=30$.

  \item {\tt svaria\_SampleProd} with $N=1$, $n=10^7$, $r=0$, $t=10$.

  \item {\tt svaria\_SampleProd} with $N=1$, $n=10^7$, $r=0$, $t=30$.

  \item {\tt svaria\_SampleMean} with $N=10^7$, $n=20$, $r=0$.

  \item {\tt svaria\_SampleCorr} with $N=1$, $n=5*10^8$, $r=0$, $k=1$.

  \item {\tt svaria\_AppearanceSpacings} with $N=1$,  $Q=10^7$, $K=4*10^8$,
   $r=0$,  $s=30$, $L=15$.

  \item {\tt svaria\_AppearanceSpacings} with $N=1$,  $Q=10^7$, $K=10^8$,
   $r=20$,  $s=10$, $L=15$.

  \item {\tt svaria\_WeightDistrib} with $N=1$, $n=2*10^6$, $r=0$, $k=256$,
   {\tt Alpha} $=0$,  {\tt Beta} $=1/8$.

  \item {\tt svaria\_WeightDistrib} with $N=1$, $n=2*10^6$, $r=8$, $k=256$,
   {\tt Alpha} $=0$,  {\tt Beta} $=1/8$.

  \item {\tt svaria\_WeightDistrib} with $N=1$, $n=2*10^6$, $r=16$, $k=256$,
   {\tt Alpha} $=0$,  {\tt Beta} $=1/8$.

  \item {\tt svaria\_WeightDistrib} with $N=1$, $n=2*10^6$, $r=24$, $k=256$,
   {\tt Alpha} $=0$,  {\tt Beta} $=1/8$.

  \item {\tt svaria\_SumCollector} with $N=1$, $n=2*10^7$, $r=0$,  $g=10$.

  \item {\tt smarsa\_MatrixRank} with $N=1$, $n=10^6$, $r=0$,
   $s=30$, $L=k=60$.

  \item {\tt smarsa\_MatrixRank} with $N=1$, $n=10^6$, $r=20$,
   $s=10$, $L=k=60$.

  \item {\tt smarsa\_MatrixRank} with $N=1$, $n=50000$, $r=0$,
   $s=30$, $L=k=300$.

  \item {\tt smarsa\_MatrixRank} with $N=1$, $n=50000$, $r=20$,
   $s=10$, $L=k=300$.

  \item {\tt smarsa\_MatrixRank} with $N=1$, $n=2000$, $r=0$,
   $s=30$, $L=k=1200$.

  \item {\tt smarsa\_MatrixRank} with $N=1$, $n=2000$, $r=20$,
   $s=10$, $L=k=1200$.

  \item {\tt smarsa\_Savir2} with $N=1$, $n=2*10^7$, $r=0$, $m=2^{20}$, $t=30$.

  \item {\tt smarsa\_GCD} with $N=1$, $n=10^8$, $r=0$, $s=30$.

  \item {\tt smarsa\_GCD} with $N=1$, $n=4*10^7$, $r=10$, $s=20$.

  \item {\tt swalk\_RandomWalk1} with $N=1$, $n=5*10^7$, $r=0$,
   $s=30$, $L_0=L_1=90$.

  \item {\tt swalk\_RandomWalk1} with $N=1$, $n=10^7$, $r=20$,
   $s=10$, $L_0=L_1=90$.

  \item {\tt swalk\_RandomWalk1} with $N=1$, $n=5*10^6$, $r=0$,
   $s=30$, $L_0=L_1=1000$.

  \item {\tt swalk\_RandomWalk1} with $N=1$, $n=10^6$, $r=20$,
   $s=10$, $L_0=L_1=1000$.

  \item {\tt swalk\_RandomWalk1} with $N=1$, $n=5*10^5$, $r=0$,
   $s=30$, $L_0=L_1=10000$.

  \item {\tt swalk\_RandomWalk1} with $N=1$, $n=10^5$, $r=20$,
   $s=10$, $L_0=L_1=10000$.

  \item {\tt scomp\_LinearComp} with $N=1$, $n=120000$, $r=0$, $s=1$.

  \item {\tt scomp\_LinearComp} with $N=1$, $n=120000$, $r=29$, $s=1$.

  \item {\tt scomp\_LempelZiv} with $N=10$, $k=25$, $r=0$, $s=30$.

  \item {\tt sspectral\_Fourier3} with $N=50000$, $k=14$, $r=0$, $s=30$.

  \item {\tt sspectral\_Fourier3} with $N=50000$, $k=14$, $r=20$, $s=10$.

  \item {\tt sstring\_LongestHeadRun} with $N=1$, $n=1000$, $r=0$,
   $s=30$, $L=10^7$.

  \item {\tt sstring\_LongestHeadRun} with $N=1$, $n=300$, $r=20$,
   $s=10$, $L=10^7$.

  \item {\tt sstring\_PeriodsInStrings} with $N=1$, $n=3*10^8$, $r=0$,
   $s=30$.

  \item {\tt sstring\_PeriodsInStrings} with $N=1$, $n=3*10^8$, $r=15$,
   $s=15$.

  \item {\tt sstring\_HammingWeight2} with $N=100$, $n=10^8$, $r=0$,
   $s=30$, $L=10^6$.

  \item {\tt sstring\_HammingWeight2} with $N=30$, $n=10^8$, $r=20$,
   $s=10$, $L=10^6$.

  \item {\tt sstring\_HammingCorr} with $N=1$, $n=5*10^8$, $r=0$,
   $s=30$, $L=30$.

  \item {\tt sstring\_HammingCorr} with $N=1$, $n=5*10^7$, $r=0$,
   $s=30$, $L=300$.

  \item {\tt sstring\_HammingCorr} with $N=1$, $n=10^7$, $r=0$,
   $s=30$, $L=1200$.

  \item {\tt sstring\_HammingIndep} with $N=1$, $n=3*10^8$, $r=0$,
   $s=30$, $L=30$, $d=0$.

  \item {\tt sstring\_HammingIndep} with $N=1$, $n=10^8$, $r=20$,
   $s=10$, $L=30$, $d=0$.

  \item {\tt sstring\_HammingIndep} with $N=1$, $n=3*10^7$, $r=0$,
   $s=30$, $L=300$, $d=0$.

  \item {\tt sstring\_HammingIndep} with $N=1$, $n=10^7$, $r=20$,
   $s=10$, $L=300$, $d=0$.

  \item {\tt sstring\_HammingIndep} with $N=1$, $n=10^7$, $r=0$,
   $s=30$, $L=1200$, $d=0$.

  \item {\tt sstring\_HammingIndep} with $N=1$, $n=10^6$, $r=20$,
   $s=10$, $L=1200$, $d=0$.

  \item {\tt sstring\_Run} with $N=1$, $n=10^9$, $r=0$, $s=30$.

  \item {\tt sstring\_Run} with $N=1$, $n=10^9$, $r=20$, $s=10$.

  \item {\tt sstring\_AutoCor} with $N=10$, $n=10^9$, $r=0$, $s=30$,
  $d=1$.

  \item {\tt sstring\_AutoCor} with $N=5$, $n=10^9$, $r=20$, $s=10$,
  $d=1$.

  \item {\tt sstring\_AutoCor} with $N=10$, $n=10^9$, $r=0$, $s=30$,
  $d=30$.

  \item {\tt sstring\_AutoCor} with $N=5$, $n=10^9$, $r=20$, $s=10$,
  $d=10$.

\end{enumerate}

\code


void bbattery_RepeatCrush (unif01_Gen *gen, int rep[]);
\endcode
  \tab Similar to {\tt bbattery\_RepeatSmallCrush} above but applied on
  {\tt Crush}.
  \endtab
\bigskip

\hrule
\code


void bbattery_BigCrush (unif01_Gen *gen);
\endcode
  \tab
  Applies the battery {\tt BigCrush}, a suite of very stringent statistical
  tests, to the generator {\tt gen}.
  \index{BigCrush}\index{battery of tests!BigCrush}%
  Some of these tests assume that {\tt gen} returns at least 30 bits of
  resolution; if that is not the case, then the generator will certainly
  fail these particular tests. One test requires 31 bits of resolution:
  the {\tt BirthdaySpacings} test with $t=2$.
  On a PC with an AMD Athlon 64 Processor 4000+
  of clock speed 2400 MHz running with Linux, BigCrush
  will take around 8 hours of CPU time. {\tt BigCrush} uses close to
  $2^{38}$ random numbers.
  The following tests are applied:\label{bat:BigCrush}
  \endtab

  \begin{enumerate}
  \item {\tt smarsa\_SerialOver} with $N=1$,  $n=10^9$,  $r=0$,
  $d = 2^{8}$, $t=3$.

  \item {\tt smarsa\_SerialOver} with $N=1$,  $n=10^9$,  $r=22$,
  $d = 2^{8}$, $t=3$.

  \item {\tt smarsa\_CollisionOver} with $N=30$,  $n=2*10^7$,  $r=0$,
  $d = 2^{21}$, $t=2$.

  \item {\tt smarsa\_CollisionOver} with $N=30$,  $n=2*10^7$,  $r=9$,
  $d = 2^{21}$,  $t=2$.

  \item {\tt smarsa\_CollisionOver} with $N=30$,  $n=2*10^7$,  $r=0$,
  $d = 2^{14}$, $t=3$.

  \item {\tt smarsa\_CollisionOver} with $N=30$,  $n=2*10^7$,  $r=16$,
  $d = 2^{14}$,  $t=3$.

  \item {\tt smarsa\_CollisionOver} with $N=30$,  $n=2*10^7$,  $r=0$,
  $d = 64$, $t=7$.

  \item {\tt smarsa\_CollisionOver} with $N=30$,  $n=2*10^7$,  $r=24$,
  $d = 64$,  $t=7$.

  \item {\tt smarsa\_CollisionOver} with $N=30$,  $n=2*10^7$,  $r=0$,
  $d = 8$, $t=14$.

  \item {\tt smarsa\_CollisionOver} with $N=30$,  $n=2*10^7$,  $r=27$,
  $d = 8$,  $t=14$.

  \item {\tt smarsa\_CollisionOver} with $N=30$,  $n=2*10^7$,  $r=0$,
  $d =4$,  $t=21$.

  \item {\tt smarsa\_CollisionOver} with $N=30$,  $n=2*10^7$,  $r=28$,
  $d =4$,  $t=21$.

  \item {\tt smarsa\_BirthdaySpacings}  with $N=100$,  $n=10^7$, $r=0$,
   $d = 2^{31}$, $t=2$, $p=1$.

  \item {\tt smarsa\_BirthdaySpacings}  with $N=20$,  $n=2*10^7$, $r=0$,
   $d = 2^{21}$, $t=3$, $p=1$.

  \item {\tt smarsa\_BirthdaySpacings}  with $N=20$,  $n=3*10^7$, $r=14$,
   $d = 2^{16}$, $t=4$, $p=1$.

  \item {\tt smarsa\_BirthdaySpacings}  with $N=20$,  $n=2*10^7$, $r=0$,
   $d = 2^{9}$, $t=7$, $p=1$.

  \item {\tt smarsa\_BirthdaySpacings}  with $N=20$,  $n=2*10^7$, $r=7$,
   $d = 2^{9}$, $t=7$, $p=1$.

  \item {\tt smarsa\_BirthdaySpacings}  with $N=20$,  $n=3*10^7$, $r=14$,
   $d = 2^{8}$, $t=8$, $p=1$.

  \item {\tt smarsa\_BirthdaySpacings}  with $N=20$,  $n=3*10^7$, $r=22$,
   $d = 2^{8}$, $t=8$, $p=1$.

  \item {\tt smarsa\_BirthdaySpacings}  with $N=20$,  $n=3*10^7$, $r=0$,
   $d = 2^{4}$, $t=16$, $p=1$.

  \item {\tt smarsa\_BirthdaySpacings}  with $N=20$,  $n=3*10^7$, $r=26$,
   $d = 2^{4}$, $t=16$, $p=1$.

  \item {\tt snpair\_ClosePairs}  with $N=30$,  $n=6*10^6$, $r=0$, $t=3$,
   $p=0$,  $m=30$.

  \item {\tt snpair\_ClosePairs}  with $N=20$,  $n=4*10^6$, $r=0$, $t=5$,
   $p=0$,  $m=30$.

  \item {\tt snpair\_ClosePairs}  with $N=10$,  $n=3*10^6$, $r=0$, $t=9$,
   $p=0$,  $m=30$.

  \item {\tt snpair\_ClosePairs}  with $N=5$,  $n=2*10^6$, $r=0$, $t=16$,
   $p=0$,  $m=30$.

  \item {\tt sknuth\_SimpPoker} with $N=1$,  $n=4*10^8$,  $r=0$, $d=8$, $k=8$.

  \item {\tt sknuth\_SimpPoker} with $N=1$,  $n=4*10^8$,  $r=27$, $d=8$, $k=8$.

  \item {\tt sknuth\_SimpPoker} with $N=1$,  $n=10^8$,  $r=0$, $d=32$, $k=32$.

  \item {\tt sknuth\_SimpPoker} with $N=1$,  $n=10^8$,  $r=25$, $d=32$, $k=32$.

  \item {\tt sknuth\_CouponCollector} with $N=1$, $n=2*10^8$, $r=0$, $d=8$.

  \item {\tt sknuth\_CouponCollector} with $N=1$, $n=2*10^8$, $r=10$, $d=8$.

  \item {\tt sknuth\_CouponCollector} with $N=1$, $n=2*10^8$, $r=20$, $d=8$.

  \item {\tt sknuth\_CouponCollector} with $N=1$, $n=2*10^8$, $r=27$, $d=8$.

  \item {\tt sknuth\_Gap} with $N=1$, $n=5*10^8$, $r=0$, {\tt Alpha} $=0$,
    {\tt Beta} $=1/16$.

  \item {\tt sknuth\_Gap} with $N=1$, $n=3*10^8$, $r=25$, {\tt Alpha} $=0$,
    {\tt Beta} $=1/32$.

  \item {\tt sknuth\_Gap} with $N=1$, $n=10^8$, $r=0$, {\tt Alpha} $=0$,
    {\tt Beta} $=1/128$.

  \item {\tt sknuth\_Gap} with $N=1$, $n=10^7$, $r=20$, {\tt Alpha} $=0$,
   {\tt Beta} $=1/1024$.

  \item {\tt sknuth\_Run}  with $N=5$, $n=10^9$, $r=0$, {\tt Up = FALSE}.

  \item {\tt sknuth\_Run}  with $N=5$, $n=10^9$, $r=15$, {\tt Up = TRUE}.

  \item {\tt sknuth\_Permutation} with $N=1$, $n=10^9$, $r=0$, $t=3$.

  \item {\tt sknuth\_Permutation} with $N=1$, $n=10^9$, $r=0$, $t=5$.

  \item {\tt sknuth\_Permutation} with $N=1$, $n=5*10^8$, $r=0$, $t=7$.

  \item {\tt sknuth\_Permutation} with $N=1$, $n=5*10^8$, $r=10$, $t=10$.

  \item {\tt sknuth\_CollisionPermut} with $N=20$, $n=2*10^7$, $r=0$, $t=14$.

  \item {\tt sknuth\_CollisionPermut} with $N=20$, $n=2*10^7$, $r=10$, $t=14$.

  \item {\tt sknuth\_MaxOft} with $N=40$, $n=10^7$, $r=0$, $d=10^5$, $t=8$.

  \item {\tt sknuth\_MaxOft} with $N=30$, $n=10^7$, $r=0$, $d=10^5$, $t=16$.

  \item {\tt sknuth\_MaxOft} with $N=20$, $n=10^7$, $r=0$, $d=10^5$, $t=24$.

  \item {\tt sknuth\_MaxOft} with $N=20$, $n=10^7$, $r=0$, $d=10^5$, $t=32$.

  \item {\tt svaria\_SampleProd} with $N=40$, $n=10^7$, $r=0$, $t=8$.

  \item {\tt svaria\_SampleProd} with $N=20$, $n=10^7$, $r=0$, $t=16$.

  \item {\tt svaria\_SampleProd} with $N=20$, $n=10^7$, $r=0$, $t=24$.

  \item {\tt svaria\_SampleMean} with $N=2*10^7$, $n=30$, $r=0$.

  \item {\tt svaria\_SampleMean} with $N=2*10^7$, $n=30$, $r=10$.

  \item {\tt svaria\_SampleCorr} with $N=1$, $n=2*10^9$, $r=0$, $k=1$.

  \item {\tt svaria\_SampleCorr} with $N=1$, $n=2*10^9$, $r=0$, $k=2$.

  \item {\tt svaria\_AppearanceSpacings} with $N=1$,  $Q=10^7$, $K=10^9$,
   $r=0$,  $s=3$, $L=15$.

  \item {\tt svaria\_AppearanceSpacings} with $N=1$,  $Q=10^7$, $K=10^9$,
   $r=27$,  $s=3$, $L=15$.

  \item {\tt svaria\_WeightDistrib} with $N=1$, $n=2*10^7$, $r=0$, $k=256$,
   {\tt Alpha} $=0$,  {\tt Beta} $=1/4$.

  \item {\tt svaria\_WeightDistrib} with $N=1$, $n=2*10^7$, $r=20$, $k=256$,
   {\tt Alpha} $=0$,  {\tt Beta} $=1/4$.

  \item {\tt svaria\_WeightDistrib} with $N=1$, $n=2*10^7$, $r=28$, $k=256$,
   {\tt Alpha} $=0$,  {\tt Beta} $=1/4$.

  \item {\tt svaria\_WeightDistrib} with $N=1$, $n=2*10^7$, $r=0$, $k=256$,
   {\tt Alpha} $=0$,  {\tt Beta} $=1/16$.

  \item {\tt svaria\_WeightDistrib} with $N=1$, $n=2*10^7$, $r=10$, $k=256$,
   {\tt Alpha} $=0$,  {\tt Beta} $=1/16$.

  \item {\tt svaria\_WeightDistrib} with $N=1$, $n=2*10^7$, $r=26$, $k=256$,
   {\tt Alpha} $=0$,  {\tt Beta} $=1/16$.

  \item {\tt svaria\_SumCollector} with $N=1$, $n=5*10^8$, $r=0$,  $g=10$.

  \item {\tt smarsa\_MatrixRank} with $N=10$, $n=10^6$, $r=0$,
   $s=5$, $L=k=30$.

  \item {\tt smarsa\_MatrixRank} with $N=10$, $n=10^6$, $r=25$,
   $s=5$, $L=k=30$.

  \item {\tt smarsa\_MatrixRank} with $N=1$, $n=5000$, $r=0$,
   $s=4$, $L=k=1000$.

  \item {\tt smarsa\_MatrixRank} with $N=1$, $n=5000$, $r=26$,
   $s=4$, $L=k=1000$.

  \item {\tt smarsa\_MatrixRank} with $N=1$, $n=80$, $r=15$,
   $s=15$, $L=k=5000$.

  \item {\tt smarsa\_MatrixRank} with $N=1$, $n=80$, $r=0$,
   $s=30$, $L=k=5000$.

  \item {\tt smarsa\_Savir2} with $N=10$, $n=10^7$, $r=10$, $m=2^{20}$, $t=30$.

  \item {\tt smarsa\_GCD} with $N=10$, $n=5*10^7$, $r=0$, $s=30$.

  \item {\tt swalk\_RandomWalk1} with $N=1$, $n=10^8$, $r=0$,
   $s=5$, $L_0=L_1=50$.

  \item {\tt swalk\_RandomWalk1} with $N=1$, $n=10^8$, $r=25$,
   $s=5$, $L_0=L_1=50$.

  \item {\tt swalk\_RandomWalk1} with $N=1$, $n=10^7$, $r=0$,
   $s=10$, $L_0=L_1=1000$.

  \item {\tt swalk\_RandomWalk1} with $N=1$, $n=10^7$, $r=20$,
   $s=10$, $L_0=L_1=1000$.

  \item {\tt swalk\_RandomWalk1} with $N=1$, $n=10^6$, $r=0$,
   $s=15$, $L_0=L_1=10000$.

  \item {\tt swalk\_RandomWalk1} with $N=1$, $n=10^6$, $r=15$,
   $s=15$, $L_0=L_1=10000$.

  \item {\tt scomp\_LinearComp} with $N=1$, $n=400000$, $r=0$, $s=1$.

  \item {\tt scomp\_LinearComp} with $N=1$, $n=400000$, $r=29$, $s=1$.

  \item {\tt scomp\_LempelZiv} with $N=10$, $k=27$, $r=0$, $s=30$.

  \item {\tt scomp\_LempelZiv} with $N=10$, $k=27$, $r=15$, $s=15$.

  \item {\tt sspectral\_Fourier3} with $N=100000$, $r=0$, $s=3$, $k=14$.

  \item {\tt sspectral\_Fourier3} with $N=100000$, $r=27$, $s=3$, $k=14$.

  \item {\tt sstring\_LongestHeadRun} with $N=1$, $n=1000$, $r=0$,
   $s=3$, $L=10^7$.

  \item {\tt sstring\_LongestHeadRun} with $N=1$, $n=1000$, $r=27$,
   $s=3$, $L=10^7$.

  \item {\tt sstring\_PeriodsInStrings} with $N=10$, $n=5*10^8$, $r=0$,
   $s=10$.

  \item {\tt sstring\_PeriodsInStrings} with $N=10$, $n=5*10^8$, $r=20$,
   $s=10$.

  \item {\tt sstring\_HammingWeight2} with $N=10$, $n=10^9$, $r=0$,
   $s=3$, $L=10^6$.

  \item {\tt sstring\_HammingWeight2} with $N=10$, $n=10^9$, $r=27$,
   $s=3$, $L=10^6$.

  \item {\tt sstring\_HammingCorr} with $N=1$, $n=10^9$, $r=10$,
   $s=10$, $L=30$.

  \item {\tt sstring\_HammingCorr} with $N=1$, $n=10^8$, $r=10$,
   $s=10$, $L=300$.

  \item {\tt sstring\_HammingCorr} with $N=1$, $n=10^8$, $r=10$,
   $s=10$, $L=1200$.

  \item {\tt sstring\_HammingIndep} with $N=10$, $n=3*10^7$, $r=0$,
   $s=3$, $L=30$, $d=0$.

  \item {\tt sstring\_HammingIndep} with $N=10$, $n=3*10^7$, $r=27$,
   $s=3$, $L=30$, $d=0$.

  \item {\tt sstring\_HammingIndep} with $N=1$, $n=3*10^7$, $r=0$,
   $s=4$, $L=300$, $d=0$.

  \item {\tt sstring\_HammingIndep} with $N=1$, $n=3*10^7$, $r=26$,
   $s=4$, $L=300$, $d=0$.

  \item {\tt sstring\_HammingIndep} with $N=1$, $n=10^7$, $r=0$,
   $s=5$, $L=1200$, $d=0$.

  \item {\tt sstring\_HammingIndep} with $N=1$, $n=10^7$, $r=25$,
   $s=5$, $L=1200$, $d=0$.

  \item {\tt sstring\_Run}  with $N=1$, $n=2*10^9$, $r=0$, $s=3$.

  \item {\tt sstring\_Run}  with $N=1$, $n=2*10^9$, $r=27$, $s=3$.

  \item {\tt sstring\_AutoCor} with $N=10$, $n=10^9$, $r=0$, $s=3$,
   $d=1$.

  \item {\tt sstring\_AutoCor} with $N=10$, $n=10^9$, $r=0$, $s=3$,
   $d=3$.

  \item {\tt sstring\_AutoCor} with $N=10$, $n=10^9$, $r=27$, $s=3$,
   $d=1$.

  \item {\tt sstring\_AutoCor} with $N=10$, $n=10^9$, $r=27$, $s=3$,
   $d=3$.

\end{enumerate}

\code


void bbattery_RepeatBigCrush (unif01_Gen *gen, int rep[]);
\endcode
  \tab Similar to {\tt bbattery\_RepeatSmallCrush} above but applied on
  {\tt BigCrush}.
  \endtab
\bigskip


\bigskip
\hrule
\code


void bbattery_Rabbit (unif01_Gen *gen, double nb);
\endcode
  \tab Applies the {\tt Rabbit} battery of tests to the generator {\tt gen}
   using at most {\tt nb} bits for each test. See the description of the
   tests in {\tt bbattery\_RabbitFile}.
  \endtab
\code

void bbattery_RabbitFile (char *filename, double nb);
\endcode
  \tab Applies the {\tt Rabbit} battery of tests to the first {\tt nb}
   bits (or less, if {\tt nb} is too large) of the binary file
   {\tt filename}.
  \index{Rabbit}\index{battery of tests!Rabbit}%
%  All the following tests are applied on the same sequence of
%  bits of the file.
  For each test, the file is reset and the test is applied to the bit
  stream starting at the beginning of the file. The bits themselves are
  processed in nearly all the tests as blocks of 32 bits (unsigned
  integers); the two exceptions are {\tt svaria\_AppearanceSpacings},
  which uses blocks of 30 bits (and discards the last 2 bits out of
  each block of 32),
  and {\tt sstring\_PeriodsInStrings} which uses blocks of 31 bits (and
  discards 1 bit out of 32).
  The parameters of each test are chosen automatically as a function of
  {\tt nb}, in order to make the test reasonably sensitive.
% Constraints are also set on the parameter $n$, because of memory limitations,
% in such a way that no test should normally require more than 420 Megs of memory.
%  Thus  when {\tt nb} is not too large, the number of replications $N$ is
%   chosen equal to 1, and the sample size $n$ is determined as a function of
%   {\tt nb}.
  On a PC with an Athlon processor of clock speed
  1733 MHz running under Linux,  {\tt Rabbit} will take about 5 seconds to
  test a stream of $2^{20}$ bits, 90 seconds to test a stream of
  $2^{25}$ bits, and 28 minutes to test a stream of
  $2^{30}$ bits.
  Restriction: {\tt nb} $\ge 500$.
  \endtab

\begin{enumerate}
  \item {\tt smultin\_MultinomialBitsOver}
  \item {\tt snpair\_ClosePairsBitMatch} in $t=2$ dimensions.
  \item {\tt snpair\_ClosePairsBitMatch} in $t=4$ dimensions.
  \item {\tt svaria\_AppearanceSpacings}
  \item {\tt scomp\_LinearComp}
  \item {\tt scomp\_LempelZiv}
  \item {\tt sspectral\_Fourier1}
  \item {\tt sspectral\_Fourier3}
  \item {\tt sstring\_LongestHeadRun}
  \item {\tt sstring\_PeriodsInStrings}
  \item {\tt sstring\_HammingWeight} with blocks of $L = 32$ bits.
  \item {\tt sstring\_HammingCorr} with blocks of $L = 32$ bits.
  \item {\tt sstring\_HammingCorr} with blocks of $L = 64$ bits.
  \item {\tt sstring\_HammingCorr} with blocks of $L = 128$ bits.
  \item {\tt sstring\_HammingIndep} with blocks of $L = 16$ bits.
  \item {\tt sstring\_HammingIndep} with blocks of $L = 32$ bits.
  \item {\tt sstring\_HammingIndep} with blocks of $L = 64$ bits.
  \item {\tt sstring\_AutoCor} with a lag $d = 1$.
  \item {\tt sstring\_AutoCor} with a lag $d = 2$.
  \item {\tt sstring\_Run}
  \item {\tt smarsa\_MatrixRank} with $32 \times 32$ matrices.
  \item {\tt smarsa\_MatrixRank} with $320 \times 320$ matrices.
  \item {\tt smarsa\_MatrixRank} with $1024 \times 1024$ matrices.
  \item {\tt swalk\_RandomWalk1} with walks of length $L = 128$.
  \item {\tt swalk\_RandomWalk1} with walks of length $L = 1024$.
  \item {\tt swalk\_RandomWalk1} with walks of length $L = 10016$.
\end{enumerate}

\code

void bbattery_RepeatRabbit (unif01_Gen *gen, double nb, int rep[]);
\endcode
  \tab Similar to {\tt bbattery\_RepeatSmallCrush} above but applied on
 {\tt Rabbit}.
  \endtab

\bigskip
\hrule
\code


void bbattery_Alphabit (unif01_Gen *gen, double nb, int r, int s);
\endcode
  \tab Applies the {\tt Alphabit} battery of tests to the generator {\tt gen}
   using at most {\tt nb} bits for each test. The bits themselves are
  processed as blocks of 32 bits (unsigned integers). For each block of
  32 bits, the $r$ most significant bits are dropped, and the test is
  applied on the $s$ following bits. If one wants to test all bits of
  the stream, one should set $r=0$ and $s=32$. If one wants to test only
  1 bit out of 32, one should set $s=1$.
   See the description of the tests in {\tt bbattery\_AlphabitFile}.
  \endtab
\code


void bbattery_AlphabitFile (char *filename, double nb);
\endcode
  \tab Applies the {\tt Alphabit} battery of tests to the first {\tt nb}
   bits (or less, if {\tt nb} is too large) of the binary file
   {\tt filename}. Unlike the {\tt bbattery\_Alphabit} function above,
  \index{Alphabit}\index{battery of tests!Alphabit}%
%%   all the following tests are applied on the same sequence of
%%  bits of the file, as the binary file is rewound
  for each test, the file is rewound and the test is applied to the bit
  stream starting at the beginning of the file.
  On a PC with an Athlon processor of clock speed 1733 MHz running under
  Linux,  {\tt Alphabit} takes about 4.2 seconds to test a file of
  $2^{25}$ bits, and 2.3 minutes to test a file of $2^{30}$ bits.

  {\tt Alphabit} and {\tt AlphabitFile} have been designed primarily to test
   {\it hardware} random bits generators. The four {\tt MultinomialBitsOver}
  tests should detect correlations between successive bits by
   applying a {\tt SerialOver} test to overlapping blocks of $2$, $4$,
   $8$ and $16$ bits. The {\tt Hamming} tests should detect correlations
  between the successive bits of overlapping blocks of $16$ and $32$ bits,
  and the {\tt RandomWalk} tests consider blocks of 64 and 320 bits.

%  Restriction: {\tt nb} $\ge 500$.
  \endtab

\begin{enumerate}
  \item {\tt smultin\_MultinomialBitsOver} with $L=2$.
  \item {\tt smultin\_MultinomialBitsOver} with $L=4$.
  \item {\tt smultin\_MultinomialBitsOver} with $L=8$.
  \item {\tt smultin\_MultinomialBitsOver} with $L=16$.
  \item {\tt sstring\_HammingIndep} with blocks of $L = 16$ bits.
  \item {\tt sstring\_HammingIndep} with blocks of $L = 32$ bits.
  \item {\tt sstring\_HammingCorr} with blocks of $L = 32$ bits.
  \item {\tt swalk\_RandomWalk1} with walks of length $L = 64$.
  \item {\tt swalk\_RandomWalk1} with walks of length $L = 320$.
\end{enumerate}

\code

void bbattery_RepeatAlphabit (unif01_Gen *gen, double nb, int r, int s,
                              int rep[]);
\endcode
  \tab Similar to {\tt bbattery\_RepeatSmallCrush} above but applied on
  {\tt Alphabit}.
  \endtab
\code


void bbattery_BlockAlphabit (unif01_Gen *gen, double nb, int r, int s);
void bbattery_BlockAlphabitFile (char *filename, double nb);
\endcode
 \tab Apply the {\tt Alphabit} battery of tests repeatedly to the generator
  {\tt gen} or to the binary file {\tt filename} after reordering the bits
  as described in the filter {\tt unif01\_CreateBitBlockGen}.
  {\tt Alphabit} will be applied for the different values of
   $w \in \{1, 2, 4, 8, 16, 32\}$. If $s <32$, only values of $w \le s$ will
   be used. Each test uses at most {\tt nb} bits.
   See the description of the tests in {\tt bbattery\_AlphabitFile}.
  \endtab
\code


void bbattery_RepeatBlockAlphabit (unif01_Gen *gen, double nb, int r, int s,
                                   int rep[], int w);
\endcode
  \tab Similar to {\tt bbattery\_RepeatSmallCrush} above but applied on
   {\tt BlockAlphabit}. The parameter $w$ is the one described in
    {\tt bbattery\_BlockAlphabit}.  Restrictions:
   $w \in \{1, 2, 4, 8, 16, 32\}$ and $w \le s$.
  \endtab




\guisec {Other Tests Suites}
\code


void bbattery_pseudoDIEHARD (unif01_Gen *gen);
\endcode
  \tab
 \index{DIEHARD}\index{battery of tests!PseudoDIEHARD}%
 \index{battery of tests!DIEHARD}%
  Applies the battery PseudoDIEHARD, which implements most of
  the tests in the popular battery DIEHARD \cite{rMAR96b}
  or, in some cases, close approximations to them. \textbf{We do not recommend
  this battery as it is not very stringent} (we do not know of any generator
  that passes the batteries \textsc{Crush} and  \textsc{BigCrush}, and
  fails PseudoDIEHARD, while we have seen the converse for several defective
  generators). It is included here only for convenience to the user.
% Some of the tests in DIEHARD however are not implemented in TestU01.
  The DIEHARD tests and the corresponding tests in  PseudoDIEHARD are:
  \endtab

\begin{enumerate}

\item The {\bf Birthday Spacings} test. This corresponds
  to {\tt smarsa\_BirthdaySpacings} with $n=512$, $d=2^{24}$, $t=1$ and
  $r=0, 1, 2, 3, 4, 5, 6, 7, 8, 9$ successively. The test with each
 value of $r$ is repeated 500 times and a chi-square test is then applied.

\item The {\bf Overlapping 5-Permutation} test. This test is not
implemented in TestU01.

\item The {\bf Binary Rank Tests for Matrices}. This corresponds
 to {\tt smarsa\_MatrixRank}.

\item The {\bf Bitstream} test.  Closely related to
 {\tt smultin\_MultinomialBitsOver} with  {\tt Delta} $= -1$,
 $n=2^{21}$, $L=20$.

\item The {\bf OPSO} test. This corresponds to {\tt smarsa\_CollisionOver}
 with $n=2^{21}$, $d=1024$, $t=2$ and all values of $r$ from 0 to 22.

\item The {\bf OQSO} test. This corresponds to {\tt smarsa\_CollisionOver}
 with $n=2^{21}$, $d=32$, $t=4$ and all values of $r$ from 0 to 27.

\item The {\bf DNA} test. This corresponds to {\tt smarsa\_CollisionOver}
 with $n=2^{21}$, $d=4$, $t=10$ and all values of $r$ from 0 to 30.

\item The {\bf Count-the-1's} test is not implemented in TestU01. It is
a 5-dimensional overlapping version of {\tt sstring\_HammingIndep}.

\item The {\bf Parking Lot} test is not implemented in TestU01.

\item The {\bf Minimum Distance} test. Closely related
  to {\tt snpair\_ClosePairs}  with
 $N=100$, $n=8000$, $t=2$, $p=2$, $m=1$.

\item The {\bf 3-D Spheres} test. Closely related
  to {\tt snpair\_ClosePairs}  with
 $N=20$, $n=4000$, $t=3$, $p=2$, $m=1$.

\item The {\bf Squeeze} test.  Closely related
  to {\tt smarsa\_Savir2}.

\item The {\bf Overlapping Sums} test  is not implemented in TestU01.

\item The {\bf Runs} test.  This corresponds to  {\tt sknuth\_Run}.

\item The {\bf Craps} test  is not implemented in TestU01.

\end{enumerate}


\bigskip
\hrule

\paragraph{The NIST test suite}

\code


void bbattery_NISTFile(char* filename, double nb);
\endcode
\tab
  Applies the {\tt NIST} battery of tests to the first {\tt nb}
  bits (or less, if {\tt nb} is too large) of the binary file
  {\tt filename}.
  The NIST (National Institute of Standards and Technology) of the U.S.
  federal government has proposed a statistical test suite \cite{rRUK01a}
  for use in the evaluation of the randomness of bitstreams
  produced by cryptographic random number generators.
  \index{NIST tests suite}\index{battery of tests!NIST}%
  %% A close equivalent to the NIST test suite is the battery
  %%  {\tt sbattery\_myNIST}. It will apply on a generator the tests in the
  %%  NIST test suite that are available in TestU01.
  The test parameters are not predetermined, we then used either the parameters
  used by other batteries or custom parameters we judged interesting.
  We will use the name coming from
  \url{https://csrc.nist.gov/Projects/Random-Bit-Generation/Documentation-and-Software/Guide-to-the-Statistical-Tests}
  and~\cite{rRUK03a}. The bits themselves are processed in nearly all the tests
  as blocks of 32 bits (unsigned integers). The following parameter sets are
  given for a 1 GiB and a 4 GiB file. We consider that 
  The NIST tests and the equivalent tests in TestU01 are:

\endtab

\begin{enumerate}
  \item The {\bf Frequency (Monobit)} test. \textit{The focus of the test is
    the proportion of zeroes and ones for the entire sequence. The purpose of
  this test is to determine whether that number of ones and zeros in a sequence
are approximately the same as would be expected for a truly random sequence. The
test assesses the closeness of the fraction of ones to $1/2$, that is, the number of
ones and zeroes in a sequence should be about the same.} This corresponds to
{\tt sstring\_HammingWeight2} with $L = n$. The code is adapted from {\tt Crush}
and {\tt BigCrush}, with $N=1$.
% NIST uses a half normal distribution of the random walk W where bit 0 is
% replaced by -1 and bit 1 by +1. W = s1 + s2 + ...sn.
% TestU01 uses a chi-square distribution with one degree of freedom
% for the number of 1 bits.

\item The test for {\bf Frequency Within A Block}. \textit{The
  focus of the test is the proportion of zeroes and ones within $L$-bit blocks.
The purpose of this test is to determine whether the frequency of ones is an
$L$-bit block is approximately $L/2$.} Corresponds to
 {\tt sstring\_HammingWeight2}, for which we choose $L = 32$ (the current block
 size). The code is adapted from {\tt Crush} and {\tt BigCrush}, with $N=1$.

\item The {\bf Runs} test. \textit{The focus of this test is the total number of
  zero and one runs in the entire sequence, where a run is an uninterrupted
sequence of identical bits. A run of length k means that a run consists of
exactly k identical bits and is bounded before and after with a bit of the
opposite value. The purpose of the runs test is to determine whether the number
of runs of ones and zeros of various lengths is as expected for a random
sequence. In particular, this test determines whether the oscillation between
such substrings is too fast or too slow..} Is implemented as {\tt sstring\_Run}.
We take the parameters of {\tt Rabbit} (we use only the first set of parameters,
the second is documented but seems not implemented in {\tt Rabbit}). Note that,
compared to the documentation of {\tt Rabbit}, we are sure that $s=32$.  The
parameters we used are then:
  \begin{itemize}
    \item $N=20$ and $n=9.981\cdot 10^8$ for a 1 GiB file and
    \item $N=7$ and $n=8.729\cdot 10^8$ for a 4 GiB file.
  \end{itemize}

\item The test for the {\bf Longest Run Of Ones In A Block}. \textit{The focus
  of the test is the longest run of ones within $L$-bit blocks. The purpose of
this test is to determine whether the length of the longest run of ones within
the tested sequence is consistent with the length of the longest run of ones
that would be expected in a random sequence. Note that an irregularity in the
expected length of the longest run of ones implies that there is also an
irregularity in the expected length of the longest run of zeroes. Long runs of
zeroes were not evaluated separately due to a concern about statistical
independence among the tests..} Is implemented as the test {\tt
sstring\_LongestHeadRun}. We take the parameters of {\tt Rabbit}, that is $N=1$,
$n=600$ and $L = 1.888\cdot10^8$ for a 1 GiB file and $L = 5.726\cdot10^7$.

\item The {\bf Random Binary Matrix Rank} test. \textit{The focus of the test is
  the rank of disjoint sub-matrices of the entire sequence. The purpose of this
test is to check for linear dependence among fixed length substrings of the
original sequence.} Is implemented as {\tt smarsa\_MatrixRank}. We take the
parameters of {\tt Rabbit}, with $32 \times 32$ matrices, with $320 \times 320$
matrices and with $1024 \times 1024$ matrices, with $N = 1$ and respectively
$n=5\cdot 10^7$, $n=2.706\cdot 10^7$ and $n=2.643\cdot10^6$ for a 1 GiB file,
and respectively $n=3.355\cdot10^7$, $n=3.355\cdot10^5$ and $n=3.276\cdot10^4$
for a 4 GiB file.

\item The {\bf Discrete Fourier Transform (Spectral)} test. \textit{The focus of
  this test is the peak heights in the discrete Fast Fourier Transform. The
purpose of this test is to detect periodic features (i.e., repetitive patterns
that are near each other) in the tested sequence that would indicate a deviation
from the assumption of randomness.} Is implemented as {\tt sspectral\_Fourier1}.
We take the parameters of {\tt Rabbit}, with $N=1$ and $k=20$ for both cases.

\item The {\bf Non-Overlapping (Aperiodic) Template Matching} test. \textit{The
  focus of this test is the number of occurrences of pre-defined target
substrings. The purpose of this test is to reject sequences that exhibit too
many occurrences of a given non-periodic (aperiodic) pattern. For this test and
for the Overlapping Template Matching test, an m-bit window is used to search
for a specific m-bit pattern. If the pattern is not found, the window slides one
bit position. For this test, when the pattern is found, the window is reset to
the bit after the found pattern, and the search resumes.} Is implemented as the
test {\tt smarsa\_CATBits}. This test is for now not available in the battery.

\item The {\bf Overlapping (Periodic) Template Matching} test. \textit{The focus
  of this test is the number of pre-defined target substrings. The purpose of
this test is to reject sequences that show deviations from the expected number
of runs of ones of a given length. Note that when there is a deviation from the
expected number of ones of a given length, there is also a deviation in the runs
of zeroes. Runs of zeroes were not evaluated separately due to a concern about
statistical independence among the tests. For this test and for the
Non-overlapping Template Matching test, an m-bit window is used to search for a
specific m-bit pattern. If the pattern is not found, the window slides one bit
position. For this test, when the pattern is found, the window again slides one
bit, and the search is resumed.} This test does not exist as such in TestU01,
but a similar and more powerful test is {\tt smultin\_MultinomialBitsOver}. This
test is for now not available in the battery.

\item The {\bf Maurer's Universal Statistical} test. \textit{The focus of this
  test is the number of bits between matching patterns. The purpose of the test
is to detect whether or not the sequence can be significantly compressed without
loss of information. An overly compressible sequence is considered to be
non-random.} This test is implemented as {\tt svaria\_Appear\-anceSpacings}. We
take the parameters of {\tt Rabbit}, with $N = 1$, $K=Q = 9.997\cdot10^6$, $L =
30$ and $s=30$ for a 1 GiB file, and $N=1$, $K=Q=9.907\cdot10^6$, $L=15$ and
$s=30$ for a 4 GiB file.

\item The {\bf Linear Complexity} test. \textit{The focus of this test is the
  length of a generating feedback register. The purpose of this test is to
determine whether or not the sequence is complex enough to be considered random.
Random sequences are characterized by a longer feedback register. A short
feedback register implies non-randomness.} Is implemented as part of {\tt
scomp\_LinearComp}. We take the parameters of {\tt Rabbit}, with $N = 1$ and
$n=3\cdot10^5$ for a both cases.

\item The {\bf Serial} test. \textit{The focus of this test is the frequency of
  each and every overlapping m-bit pattern across the entire sequence. The
purpose of this test is to determine whether the number of occurrences of the 2m
m-bit overlapping patterns is approximately the same as would be expected for a
random sequence. The pattern can overlap.} Corresponds to {\tt
smultin\_MultinomialBitsOver} with ${\tt Delta} = 1$, ${\tt bmax} = 3$. We take
the parameters $N=2772$ and $n=9.996\cdot10^8$ for a 1 GiB file and $N=35$ and
$n=9.717\cdot10^8$ for a 4 GiB from {\tt Alphabit}, with $L=2$, $L=4$, $L=8$ and $L=16$.

\item The {\bf Approximate Entropy} test. \textit{The focus of this test is the
  frequency of each and every overlapping m-bit pattern. The purpose of the test
is to compare the frequency of overlapping blocks of two consecutive/adjacent
lengths (m and m+1) against the expected result for a random sequence.}
Corresponds to {\tt smultin\_MultinomialBitsOver} with ${\tt Delta} = 0$, and to
{\tt sentrop\_EntropyDiscOver} or  {\tt sentrop\_EntropyDiscOver2}. These two
last tests are for now not available in the battery, and the first one need to
be reviewed in term of parameters.

\item The {\bf Cumulative Sum (Cusum)} test. \textit{The focus of this test is
  the maximal excursion (from zero) of the random walk defined by the cumulative
sum of adjusted $(-1, +1$) digits in the sequence. The purpose of the test is to
determine whether the cumulative sum of the partial sequences occurring in the
tested sequence is too large or too small relative to the expected behavior of
that cumulative sum for random sequences. This cumulative sum may be considered
as a random walk. For a random sequence, the random walk should be near zero.
For non-random sequences, the excursions of this random walk away from zero will
be too large.} This test is  closely related to the $M$ statistic in  {\tt
swalk\_RandomWalk1}. We take the parameters from {\tt Rabbit}.

\item The {\bf Random Excursions} test. \textit{The focus of this test is the
  number of cycles having exactly K visits in a cumulative sum random walk. The
cumulative sum random walk is found if partial sums of the (0,1) sequence are
adjusted to (-1, +1). A random excursion of a random walk consists of a sequence
of n steps of unit length taken at random that begin at and return to the
origin. The purpose of this test is to determine if the number of visits to a
state within a random walk exceeds what one would expect for a random sequence.}
This test does not exist in TestU01, but closely related tests are in {\tt
swalk\_RandomWalk1}. We take the parameters from {\tt Rabbit}.

\item The {\bf Random Excursions Variant} test. \textit{The focus of this test
  is the number of times that a particular state occurs in a cumulative sum
random walk. The purpose of this test is to detect deviations from the expected
number of occurrences of various states in the random walk.} This test does not
exist in TestU01, but a closely related test is based on the $R$ statistic in
{\tt swalk\_RandomWalk1}. We take the parameters from {\tt Rabbit}. These three
last tests are done with the parameters:

 \begin{itemize}
   \item $N=1$, $L=128$ and $n=5\cdot10^8$ for a 1 GiB file, $n=2.684\cdot10^8$
     for a 4 GiB file;
   \item $N=1$, $L=1024$ and $n=5\cdot10^7$ for a 1 GiB file, $n=3.246\cdot10^7$
     for a 4 GiB file;
   \item $N=1$, $L=10016$ and $n=5\cdot10^6$ for a 1 GiB file, $n=3.418\cdot10^6$
     for a 4 GiB file.
\end{itemize}

\end{enumerate}


\bigskip
\hrule
\code


void bbattery_FIPS_140_2 (unif01_Gen *gen);
void bbattery_FIPS_140_2File (char *filename);
\endcode
\tab
\index{FIPS-140-2}\index{battery of tests!FIPS-140-2}%
 These functions apply the four tests described in the NIST document
 {\sl FIPS PUB 140-2, Security Requirements for Cryptographic Modules},
 page 35, with exactly the same parameters (see the WEB page at
  \url{http://csrc.nist.gov/rng/rng6_3.html}).  They report the values
 of the test statistics and their $p$-values (except for the runs test)
 and indicate which values fall outside the intervals specified by
 FIPS-140-2. The first function applies the tests on a generator {\tt gen},
 and the second applies them on the file of bits {\tt filename}. First,
 20000 bits are generated and put in an array, then the tests are applied
 upon these. The tests applied are:
\endtab

\begin{enumerate}
\item The {\bf Monobit} test. This corresponds to
% {\tt sstring\_HammingWeight2} with $L = n$.
 {\tt smultin\_MultinomialBits} with $s=32$, $L=1$, $n=20000$.

\item The ``{\bf poker}'' test, which is in fact equivalent to
 {\tt smultin\_MultinomialBits} with $s=32$, $L=4$, $n=5000$.

\item The {\bf Runs} test, which is related to {\tt sstring\_Run}.

\item The test for the {\bf Longest Run of Ones in a Block},
 which is implemented as {\tt sstring\_LongestHeadRun}.
\end{enumerate}


\code
\hide
#endif
\endhide
\endcode
