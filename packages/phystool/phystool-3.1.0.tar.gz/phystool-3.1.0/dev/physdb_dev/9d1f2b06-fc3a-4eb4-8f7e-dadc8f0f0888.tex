\begin{theory}[title=Inclusion d'image]
	\verb|physool| distingue deux types d'images:
	\begin{enumerate}
		\item Celles qui sont crées en \LaTeX et gérées par \verb|phystool|
		\item et celles qui sont simplement utilisées.
	\end{enumerate}

	Dans les deux cas, il est nécessaire que \verb|phystool| connaisse le
	chemin de ces fichiers. Cela se fait au travers de la commandes:
	\begin{quote}
		\verb|\graphicspath{{\c_pdb_path_tl}, {\c_pdb_path_tl figures}}|
	\end{quote}
	Le premier chemin est celui de la base de donnée car la constante
	\verb|\c_pdb_path_tl| a été définie au travers de l'appel de
	\verb|\PdbSetDBPath|. Le deuxième chemin est celui du sous-répertoire
	\verb|figures| contenant les différentes images.

	Grâce à la définition de ces chemin, il est possible d'inclure ces deux
	types d'image facilement:
	\begin{quote}
		\verb|\includegraphics[scale=0.4]{tech_support_cheat_sheet.png}|
		\verb|\PdbTikz{01cdfb94-e20e-446c-91df-12eacc7ab474}|
	\end{quote}

	\begin{center}
		\PdbTikz{01cdfb94-e20e-446c-91df-12eacc7ab474}
		\includegraphics[scale=0.4]{tech_support_cheat_sheet.png}
	\end{center}
\end{theory}
